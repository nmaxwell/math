
\break

\begin{flushleft}
Functional analysis. \\
\end{flushleft}

\begin{flushleft}
\addvspace{5pt} \hrule
\end{flushleft}	



Definitions: \\


\noindent
For a normed space $X$, write $\Ball(X) \defeq \{  x \in X; ||x|| \le 1 \}$, and $\U_X \defeq \{ x \in X; ||X|| < 1 \}$. \\

\noindent
A function $f:X \rarw Y$, between topological spaces is continuous if $f^{-1}(U)$ is open for all open $U \subset Y$. \\

\noindent
A function $f:X \rarw Y$, between topological spaces is an open map if $f(U)$ is open for all open $U \subset X$. \\

\noindent
A homeomorphism or bicontinuous map is a bijective map which is also open, or $f^{-1}$ is also continuous. \\


Prop: for a bijective map, $f:X \rarw Y$, $f$ is an open map iff $f^{-1}$ is continuous. \\


The Baire catagory theorem: If $( A_k )_{k \in \nats}$ is a sequence of open dnse sets in a complete metric space $X$, then $\cap_{k \in \nats} A_k$ is dense in $X$. \\

\noindent
Proof: ADD \\
%Let $B_0$ be any closed ball of radius $r>0$, in $X$. If we can show $B_0 \cap \cap_{k \in \nats} A_k  \not = \phi$ then the theorem is proved. 


Corollary: A complete metric space $X$ cannod be written as a countable union of closed sets, each of which have empty interior. \\

\noindent
Proof: ADD \\

Lemma: If $T: X \rarw Y$ is a bounded operator between Banach spaces, $X,Y$, and if $r \U_Y \subset \overline{ T( \U_X) }$, then $r \U_Y \subset T(U_X)$ \\

\noindent
Proof: ADD \\

Thoerem: (The open mapping theorem) If $T: X \rarw Y$ is a surjective bounded linear operator between Banach spaces $X,Y$, then $T$ is open.\\

\noindent
Proof: ADD \\

Corollary: If $T: X \rarw Y$ is an bijective, bounded linear operator between Banach spaces, then $T$ is bicontinuous.  \\

\noindent
Proof: By the open mapping theorem, $T$ is open, and hence bicontinuous. \\

The closed graph theorem: If $T: X \rarw Y$ is a linear operator between Banach spaces then TFAE;

\begin{enumerate}[i)]
\item
$T$ is bounded. 
\item
The graph of $T$, namely $\G(T) \defeq \{  (x, T(x)); x \in X \}$, is closed in $X \times Y$, wrt to the product topology.
\item
Whenever $(X_k)_{k \in \nats} \subset X$, with $X_k \rarw x \in X$, and $T(x_k) \rarw u \in Y$, then $y = T(x)$.
\item 
Whenever  $(X_k)_{k \in \nats} \subset X$, with $X_k \rarw 0$, and $T(x_k) \rarw y \in Y$, then $y=0$.
\end{enumerate}


Theorem: ( the principle of uniform boundedness(PUB)). Suppose that $X,Y$ are Banach spaces, and $S \subset B(X,Y)$. Suppose that for every $x \in X$, $\{ Tx; \; T \in S\}$ is bounded in $Y$. Then there is a constant $M$ with $||T|| \le M$ for all $T \in S$. \\

\noindent
Proof: Let $E_n = \{ x \in X; \;  \SUP{ ||Tx||; T \in S } \le n \} = \cap_{T \in S} \{ x \in X; ||Tx|| \le n \}$.  By the Baire catagory teorem, there must exist an $n$ suc that $E_n$ has an interior point $x_0$, say. Thus there exists $r >0$ with $\overline{ B(x_0, r) } \subset E_n$, beucase if $||x|| \le r$, then $x + x_0 \in \overline{ B(x_0, r) } \subset E_n$, so $||T|| \le ||T(x+x_0)|| + ||Tx_0|| \le 2n$ for any $T \in S$. If $||x|| \le 1$ then $||r x || \le r$, so that $||T(rx)|| = r ||Tx|| \le 2n$, and $||Tx|| \le 2n/r$. Thus $||T|| \le 2n/r$. So $ \SUP{ ||T||; \; T \in S } \le 2n/r$.  \\ 








\break

Orthonormal bases and Hilbert spaces.

\vspace{0.1in}

\hrule

\vspace{0.5in}

Definition: An orthonormal set is a subset $\{ x_j; \; j\in J\}$ of a Hilbert space, if $ \langle x_i, x_j \rangle = \delta_{i,j}$ for all $i,j \in J$.  \\

\noindent
If $X$ is an orthonormal set and $J$ is finite, then for all scalars $(c_k)_{k \in J} \in \scalars$,

$$
    || \sum_{k \in J} c_k b_k || ^2 = \left \langle \sum_{j \in J} c_k x_k, \sum_{j \in J} c_k x_k \right \rangle = \sum_{i,j \in J} c_i c_j^* \langle x_i, x_j \rangle = \sum_{k \in J} |c_k|^2
$$ \\

Lemma: any orthonormal set is linearly independent. \\

\noindent
Proof: Let $X = \{ x_j; \; j\in J\}$ is an orthonormal set, $(c_k)_{k \in J} \in \scalars$. If $0 = \sum_{k \in J} c_k x_k$, then $\langle \sum_{k \in J} c_k x_k, x_j \rangle = c_j$.
































