\documentclass[12pt]{article}

\usepackage{amsmath}
\usepackage{amssymb}
\usepackage{amsfonts}
\usepackage{latexsym}
\usepackage{graphicx}

\setlength\topmargin{-1in}
\setlength{\oddsidemargin}{-0.5in}
%\setlength{\evensidemargin}{1.0in}

%\setlength{\parskip}{3pt plus 2pt}
%\setlength{\parindent}{30pt}
%\setlength{\marginparsep}{0.75cm}
%\setlength{\marginparwidth}{2.5cm}
%\setlength{\marginparpush}{1.0cm}
\setlength{\textwidth}{7.5in}
\setlength{\textheight}{10in}


\usepackage{listings}




\newcommand{\pset}[1]{ \mathcal{P}(#1) }


\newcommand{\nats}[0] { \mathbb{N}}
\newcommand{\reals}[0] { \mathbb{R}}
\newcommand{\exreals}[0] {  [-\infty,\infty] }
\newcommand{\eps}[0] {  \epsilon }
\newcommand{\A}[0] { \mathcal{A} }
\newcommand{\B}[0] { \mathcal{B} }
\newcommand{\C}[0] { \mathcal{C} }
\newcommand{\D}[0] { \mathcal{D} }
\newcommand{\E}[0] { \mathcal{E} }
\newcommand{\F}[0] { \mathcal{F} }
\newcommand{\G}[0] { \mathcal{G} }
\newcommand{\cS}[0] { \mathcal{S} }

\newcommand{\om}[0] { \omega }
\newcommand{\Om}[0] { \Omega }

\newcommand{\Bl}[0] { \mathcal{B} \ell }



\newcommand{\IF}[0] { \; \textrm{if} \; }
\newcommand{\THEN}[0] { \; \textrm{then} \; }
\newcommand{\ELSE}[0] { \; \textrm{else} \; }
\newcommand{\AND}[0]{ \; \textrm{ and } \;  }
\newcommand{\OR}[0]{ \; \textrm{ or } \; }

\newcommand{\rimply}[0] { \Rightarrow }
\newcommand{\limply}[0] { \Lefttarrow }
\newcommand{\rlimply}[0] { \Leftrightarrow }
\newcommand{\lrimply}[0] { \Leftrightarrow }

\newcommand{\rarw}[0] { \rightarrow }
\newcommand{\larw}[0] { \leftarrow }

\begin{document}

\begin{flushleft}
Research Documentation - Stochastic Differential Equations, Theory. \\
Nicholas Maxwell; Dr. Bodmann\\
\end{flushleft}

\begin{flushleft}
\addvspace{5pt} \hrule
\end{flushleft}	



%It's ok for this document to be in bad taste.

	Plagiarees: (Real Analsysis, Folland), (SDE, \O ksendal), (BM and Stochastic Calc, Karatas \& Shreve), Dr. Bodmann, Dr. Blecher, Wikipedia.


\section*{Part 1.1 - Bits from measure theory }



\begin{flushleft}
\underline{Image measures:}
\end{flushleft}

\begin{flushleft}
Given a measure space, $(X,\A,\mu)$, a measurable space, $(Y,\B)$, and an $(\A,\B)-$measurable function, $f:X \rightarrow Y$, we may construct a measure on $(Y,\B)$,  $\mu_f := \mu \circ f^{-1}$.\\
\end{flushleft}


\noindent proof:\\

(1) $\phi \in \B$, $\mu_f (\phi) = \mu(F^{-1}(\phi)) = \mu(\phi) = 0$. \\

(2) $\{ B_k \}_{k \in \nats } \in \B $, disjoint, $B := \cup_{k \in \nats} B_k$. $A := f^{-1}(B) = \cup_{k \in \nats} A_k $, $A_k := f^{-1}(B_k)$. Then $A \in \A$ by $f$ being $(\A,\B)-$measurable, and $\{ A_k \}_{k \in \nats }$ is disjoint because, when $E_1 \cap E_2 = \phi$, $E_1, E_2 \in \B$, $ \phi = f^{-1}(E_1 \cap E_2) = f^{-1}(E_1) \cap f^{-1}(E_2)$. Then, $\mu_f ( \cup_{k \in \nats} B_k )$  = $\mu( f^{-1}( \cup_{k \in \nats} B_k ) )$ = $\mu( \cup_{k \in \nats} f^{-1}( B_k ) )$ = $\mu( \cup_{k \in \nats} A_k )$ = $ \sum_{k \in \nats } \mu( A_k ) $ = $ \sum_{k \in \nats } (\mu_f)( B_k ) $, by countable additivity of $\mu$.

	
\begin{flushleft}
So $ (Y,\B,\mu_f)$ is a well defined measure space obtained from inverse images via $f$.
\end{flushleft}

\begin{flushleft}
\underline{Simple functions:}
\end{flushleft}

\begin{flushleft}
$(X,\A,\mu)$ a measuse space, $\chi_{_E}: X \rarw \{ 0, 1\}$ is the characteristic function of $E \in \pset{X}$ when $\chi_{_E}(x) = 1$ for all $x \in E$. A simple function $f: X \rarw S$ is a function which can be written as $f(x) = \sum_{k = 1}^n \, c_k \, \chi_{_{E_k}}(x)$, with $E_k \in \pset{X}, c_k \in S$. We can always choose the $ \{ E_k \}$ to be disjoint and non empty, and the $ \{ c_k \}$ to be unique and non-zero, this is called the standard representation. $f$ is measureable when the $E_k \in \A$.
\end{flushleft}


\begin{flushleft}
\underline{Definition of integral:}
\end{flushleft}

\begin{flushleft}
$(X,\A,\mu)$ a measure space. 
\end{flushleft}

1. For $f$ a positive simple function in its standard representation, define $\int_X f \, d\mu = \sum_{k=1}^n c_k \,\mu(E_k)$. \\

2. For $f$ a positive measurable function, define

$$
	\int_X f \, d\mu = \sup \{ \int_X s \, d\mu; s \textrm{ a standard simple function}, s \le f \}
$$

3. For $f$ a general measurable function, write $f = f^+ - f^-, f^+ \ge 0, f^- \ge 0$, $f^+, f^-$ are always measurable when $f$ is. Then define 

$$
\int_X f \, d\mu = \int_X f^+ \, d\mu - \int_X f^- \, d\mu 
$$

This integral exists when either of the quantities on the right hand side are finite, as $\infty - \infty$ is undefined. $f$ is integrable when its integral is finite. \\

4. If $f$ is complex valued, decompose it into real and imaginary parts, check that each is measurable.





\begin{flushleft}
\underline{A construction of integral:}
\end{flushleft}

$(X ,\A, \mu)$ a measure space. For $f: X \rarw [0, \infty]$, measurable, step two of the definition of the integral may be replaced by the following construciton. \\

Define,

$$
E_j = f^{-1} ( \, [ j 2^{-n}, (j+1) 2^{-n} ) \, ), \; \; j \in \{ 0,1,...,n2^n-1 \}
$$

$$
E_j = f^{-1} ( \, [ n, \infty ) \, ), \; \; j = n2^n
$$

$$
s_n = \sum_{j=0}^{n2^n} j2^{-n} \chi_{_{E_j}} 
$$


$E_j \in \A$ when $f$ is $( \A, \Bl(\reals^+) )$ measurabe.  \\

Clearly $s_n(x) \le n$, so then $s_n(x) \le s_{n+1}(x)$ for all $x \in X$, $n \in \nats$.\\

If at some $x \in X$, $f(x) = \infty$, then $s_n(x) = n \rarw \infty$ as $n \rarw \infty$. If at some $x \in X$, $f(x) < \infty $, then take $n$ large enough so that $f(x) \le n$, then $|s_n(x) - f(x)| \le 2^{-n} \rarw 0$ as $n \rarw \infty$. So, $s_n$ converges to $f$ pointwise. If $f$ is bounded, then the infinite case does not occur, and this convergence is uniform. \\

So, $\int_X f \, d\mu = \lim_{n \rarw \infty} \int_X s_n \, d\mu$ by Lebesgue's monotone convergence theorem.	




\section*{Part 1.2 - Borel measures in $\reals^n$}

A Borel measure is one whose domain is a borel sigma algebra. \\

Given a finite measure space, $(\reals, \Bl(\reals), \mu )$, define $F_\mu(x) = \mu((-\infty,x])$. \\

$y \ge x \rimply (-\infty,x] \subset (-\infty,y] \rimply F_\mu(x) \le F_\mu(y)$, so $F_\mu$ is monotone increasing. \\

If $x_k \rarw x$ as $k \rarw \infty$, and $x_k \ge x$, 
then $ ( -\infty, x ] = \cap_{k \in \nats} ( -\infty, x_k ] $,
so $F_\mu(x) = \mu( \cap_{k \in \nats} ( -\infty, x_k ]  )$ = $\lim_{ k \rarw \infty } \mu( ( -\infty, x_k ] )$
= $\lim_{ k \rarw \infty } F_\mu(x_k)$, when some $\mu( ( -\infty, x_k ] ) < \infty$, so $F_\mu$ is right continuous.  


ADD more detail here.


\begin{flushleft}
\underline{Theorem (Folland 1.16):}
\end{flushleft}

If $F: \reals \rarw \reals$ is any increasing and right continuous function, there is a unique Borel measure $\mu_F$ on $\reals$ such that $\mu_F((a,b]) = F(b)-F(a)$ for all $a,b$. If $G$ is another such function, we have $\mu_F = \mu_G$ iff $F-G$ is constant. Converesely, if $\mu$ is a Borel measure on $\reals$ that is finite on all bounded Borel sets, and we define


\begin{displaymath}
   F_\mu(x) = \left\{
     \begin{array}{lr}
       \mu((0,x]), &  x>0 \\
       0, & x = 0 \\
       -\mu((x,0]), & x < 0
     \end{array}
   \right.
\end{displaymath} 

 then $F_\mu$ is increasing and right continuous and $\mu = \mu_{F_\mu}$. Also, $F_\mu(x) = \mu((-\infty,x]) - \mu((-\infty,0])$, which makes sense when $\mu$ is finite. \\
 
proof: see Folland, page 35.

\begin{flushleft}
\underline{Lebesgue-Stieltjes measure:}
\end{flushleft}

$F: \reals \rarw \reals$ is any increasing and right continuous function, then

$$
	\mu(E) = \inf \left\{ \sum_{k \in \nats} ( F(b_k) - F(a_k) ); E \subset \bigcup_{ k\in \nats} (a_k,b_k] \right\}
$$

ADD more detail here.







\section*{Part 1.3 - Bits from measure theoretic probability }

\begin{flushleft}
\underline{Main idea:}
\end{flushleft}


$ (\Om, \F, P ), \;  P(\Om)=1 $ a probability space.\\


If picking $n$ points, $\{ \om_{n,k}\}_{k=1}^n$ ``at random'' from $\Om$, so all $ \om_{n,k} \in \Om$, then the following will be true

$$
	\lim_{n \rarw \infty } \frac{ \#\{ k \in \{ 1,2,...,n\} ; \om_{n,k} \in E \} }{n} = P(E), \; \textrm{for all} \; E \in \F,
$$

where $\#$ is the counting measure. \\

\begin{flushleft}
\underline{Nonsense:}
\end{flushleft}

 $\alpha:  \nats \rarw \Om$, $n \in \nats$, onto, but not one to one. Define $\#_n = \frac{1}{n} \# $, $N = \{ 1,2,...,n \}$. Then $(N, \pset{N}, \#_n )$ is a porobability space. Let $\alpha_n = \alpha |_N$, then this is an $\Om$ valued random varaible. Now we can define

$$
	P(E) = \lim_{n \rarw \infty} \#_n \alpha_n^{-1} (E), \; \textrm{for all} \; E \in \F
$$




\begin{flushleft}
\underline{Random varaibles:}
\end{flushleft}

$(S, \cS)$ a measurable space, $X: \Om \rarw S $ is called a random variable when it is $(\F, \cS)$-measurable. \\

Define $P_X: \cS \rarw [0,+\infty]$ by $P_X(E) = P( \{ \om \in \Om; X(\om) \in E \})$, this is the image measure by $X$.

$$
	P_X(S) = P( \{  \om \in \Om; X(\om) \in S \} ) = P( \Om ) = 1
$$

So the image measure induced by a random variable is a probability measure on its state space. \\

$P_X$ is called the dirstribution of $X$. \\

Define $F_X : \reals \rarw [0,1] = x \mapsto P_X((-\infty,x])$, this is called the cumulative distribution function.\\

From wikipedia:

``The probability density function of a random variable is the Radon–Nikodym derivative of the induced measure with respect to some base measure (usually the Lebesgue measure for continuous random variables).''

ADD many details here



\begin{flushleft}
\underline{Expectation:}
\end{flushleft}

Define the expectation value of $X$ as $E(X) = \int_{\Om} X \, dP$, the integral of $X$.\\

Suppose $X$ is a simple function, then $X(\om) = \sum_{k=1}^n \, c_k \chi_{_{E_k}} (\om)$, $c_k \in S$ , unique, and $E_k \in \F$ disjoint.\\

$$
	E(X) =  \sum_{k=1}^n \, c_k P(E_k)
$$

\begin{flushleft}
\underline{Discrete rv:}
\end{flushleft}

A discrete random variable $X$ is one whose state space is countable. In this case there is a bijective map, $\gamma : S \rarw \nats$, and clearly the function $\gamma \circ X	$ is $( \F, \pset{\nats} )$-measurable. \\ We may write $S = \{ x_k := \gamma^{-1}(k) \}_{k=1}^\infty$, and may define $E_k :	= X^{-1}(x_k)$, $X(\omega) = \sum_{k=1}^\infty \, x_k \chi_{_{E_k}} $. \\ If we temporarily adopt the notation ``$p(x_k) = P(X = x_k)$''$ := P(E_k)$, then


$$
	E(X) =  \sum_{k \in \nats} \, x_k p(x_k)
$$

In this simple case $\Om$ may not really be nescesary, as $( \{x_k \}, \pset{\{x_k\}}, p)$ is a probability space in it's own right, and note, with $\beta_n := X \circ \alpha_n$, $\alpha_n$ as in the above nonsense,

$$
	p(x_k) = \lim_{n \rarw \infty} \#_n \beta_n^{-1} ( x_k ), \; \textrm{for all} \; x_k
$$

%So that we can asses the random measure of a random value by making sufficiently many observations and applying the normalized counting measure.






\section*{Part 2.2 - Kolmogorov extension }

\begin{flushleft}
\underline{Notation and definitions:}
\end{flushleft}

\begin{flushleft}
Throughout, $(\Om, \F, P )$ is a probability space, $T$ is an index set, and $(S, \mathcal{S} )$ is a state space. This is cosmetic, really, $T = [0, \infty )$, time, $S = \reals^d$, $\mathcal{S} = \Bl(\reals^d)$. 
\end{flushleft}

\begin{flushleft}
We define $S^T = \{ f: T \rightarrow S \} = \prod_{t \in T} S$, and  $S^{T \times \Om} = \{ f: T \times \Om \rightarrow S \}$. 
\end{flushleft}

\begin{flushleft}
For $A$ a set, let:\\
$A^n := \{ a = (a_1, a_2, ..., a_n ) ; a_k \in A  \}$, for $n \in \nats$.\\
$\tilde{A}^n := \{ a \in A^n ; a_i \not = a_j \; \forall i \not = j  \}$, for $n \in \nats$. $\tilde{A}^n \subset A^n$ \\
$\tilde{A} := \{ a \in \tilde{A}^n; n \in \nats \}$
\end{flushleft}

\begin{flushleft}
For $A,B$ sets, $n \in \nats$, and $f: A \rarw B$ let:\\
$f(a) := ( f(a_1), f(a_2), ..., f(a_n) ) \in B^n$, for $a \in A^n$.  \\
\end{flushleft}

\begin{flushleft}
For $n \in \nats$, $t \in T^n$, and $B \in \Bl(S^n)$ define an $n$-dimensional cylinder set in $S^T$ as
$ C( B, n, t ) = \{ \om \in S^T; \om (t) \in B \} $, and then $\tilde{C} := \{ C( B, n, t ); B \in \Bl(S^n), n \in \nats, t \in T^n \}$. \\
Then let $\B(S^T)$ be the sigma algebra generated by $\tilde{C}$.
\end{flushleft}

\begin{flushleft}
\underline{Consistent family of measures:}
\end{flushleft}






\begin{flushleft}
\underline{Extension Theorem (consistency):}
\end{flushleft}










\section*{Part 3 - Brownian Motion }




\section*{Part n - Stochastic Processes }

\begin{flushleft}
\underline{Probability law of a Stochastic process:}
\end{flushleft}

%\begin{flushleft}
%Given $(\Omega,\F,P)$ a probability space, and $T$ an index set, $(S, \A)$ a measurable space. Then $X: T \times \Omega \rightarrow S$ is a stochastic process when the $t-$section of $X$, $X_t(\omega) := X(t,\omega)$ is $(\F,\A)-$measurable for all $t \in T$. Let $S^T = \{ g: T \rightarrow S \}$. \\
%Each stochastic process, $X$ induces a function, $\Phi_X: \Omega \rightarrow S^T$ by $\Phi_X(\omega) := t \mapsto X(t,\omega)$, so $\Phi_X (\omega)$ is the $\omega-$section of $X$. We're interested in defining a measure on a suitable sigma algebra on $S^T$, by pushing forward $P$ via $\Phi_X$.
%\end{flushleft}




\end{document}


















