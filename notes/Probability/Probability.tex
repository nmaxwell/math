

\documentclass[12pt]{article}

\usepackage{amsmath}
\usepackage{amssymb}
\usepackage{amsfonts}

\usepackage{latexsym}
\usepackage{graphicx}
\usepackage{colonequals}

\setlength\topmargin{-1in}
\setlength{\oddsidemargin}{-0.5in}
%\setlength{\evensidemargin}{1.0in}

%\setlength{\parskip}{3pt plus 2pt}
%\setlength{\parindent}{30pt}
%\setlength{\marginparsep}{0.75cm}
%\setlength{\marginparwidth}{2.5cm}
%\setlength{\marginparpush}{1.0cm}
\setlength{\textwidth}{7.5in}
\setlength{\textheight}{10in}


\usepackage{listings}




\newcommand{\pset}[1]{ \mathcal{P}(#1) }
\newcommand{\partset}[1]{ \mathcal{P}^{*}(#1) }
\newcommand{\st}[0]{ \textrm{ s.t. } }
\newcommand{\fall}[0] { \textrm{ for all } }
\newcommand{\wrt}[0] { \textrm{ w.r.t. } }
\newcommand{\aew}[0] { \textrm{a.e.} }

\newcommand{\nats}[0] { \mathbb{N}}
\newcommand{\reals}[0] { \mathbb{R}}
\newcommand{\cmplxs}[0] { \mathbb{C}}
\newcommand{\complexes}[0] { \mathbb{C}}
\newcommand{\exreals}[0] {  [-\infty,\infty] }
\newcommand{\eps}[0] {  \epsilon }
\newcommand{\A}[0] { \mathcal{A} }
\newcommand{\B}[0] { \mathcal{B} }
\newcommand{\C}[0] { \mathcal{C} }
\newcommand{\D}[0] { \mathcal{D} }
\newcommand{\E}[0] { \mathcal{E} }
\newcommand{\F}[0] { \mathcal{F} }
\newcommand{\G}[0] { \mathcal{G} }
\newcommand{\M}[0] { \mathcal{M} }
\newcommand{\cS}[0] { \mathcal{S} }

\newcommand{\om}[0] { \omega }
\newcommand{\Om}[0] { \Omega }

\newcommand{\Bl}[0] { \mathcal{B} \ell }

\newcommand{\Ell}[0] { \mathcal{L} }


\renewcommand{\Re}{ \operatorname{Re} }
\renewcommand{\Im}{ \operatorname{Im} }

\newcommand{\IF}[0] { \; \textrm{if} \; }
\newcommand{\THEN}[0] { \; \textrm{then} \; }
\newcommand{\ELSE}[0] { \; \textrm{else} \; }
\newcommand{\AND}[0]{ \; \textrm{ and } \;  }
\newcommand{\OR}[0]{ \; \textrm{ or } \; }

\newcommand{\rimply}[0] { \Rightarrow }
\newcommand{\limply}[0] { \Leftarrow }
\newcommand{\rlimply}[0] { \Leftrightarrow }
\newcommand{\lrimply}[0] { \Leftrightarrow }

\newcommand{\rarw}[0] { \rightarrow }
\newcommand{\larw}[0] { \leftarrow }

\newcommand{ \defeq }[0] { \colonequals }
\newcommand{ \eqdef }[0] { \equalscolon }
%\newcommand{ \cf }[1] { \chi_{_{#1}} }
\newcommand{ \cf }[1] { \mathbf{1}_{#1} }




\begin{document}

$(\Omega, \mathcal{F}, P)$ a probability space, $X: \Omega \rightarrow \mathbb{R}$, Borel measurable, and let $R = \{ X(\omega); \omega \in \Omega \}$. Let $\mu_X = E \mapsto P(X^{-1}(E))$, and assume that $\mu_X \ll \lambda$, so that $f_X := \frac{d \mu_X}{d \lambda}$ exists, and $f_X \in \mathcal{L}^1(R, \lambda)$, with $\lambda$ the Lebesgue measure on $\mathbb{R}$. Then we have $E(X) = \int_\Omega X(\omega) \,  dP(\omega) = \int_R x \, d\mu_X(x)$. Now, assume that $X \ge 0$, and $X \in \mathcal{L}^1(\lambda)$, so we can construct a sequence of partitions of $R$, $\{ E_k^n \in \mathcal{B}(\mathbb{R}); 1 \le k \le n \}$, such that

$$
E(X)  = \int_R x \, d\mu_X(x) = \lim_{n \rightarrow \infty} \sum_{k=1}^n \inf( E_k^n) \, \mu_X(E_k^n).
$$

Now, I'm going to replace $\inf( E_k^n)$ with $(x; x \in E_k^n)$, which reads, ``(an $x$ such that $x \in E_k^n$)'', in analogy to Riemann integration, where an arbitrary point may picked in the partitions, at wich to evaluate the integrand. Then also,

$$
\mu_X(E_k^n) = \lim_{N \rightarrow \infty} \frac{\#(\{ \omega_j; 1 \le j \le N, \omega_j \in \Omega, X(\omega_j) \in E_k^n \})}{N}.
$$

Combining these,


$$
E(X) =  \int_R x \, d\mu_X(x) = \lim_{n \rightarrow \infty} \frac{1}{N}  \sum_{k=1}^n (x; x \in E_k^n)  \, \lim_{N \rightarrow \infty} \#(\{ \omega_j; 1 \le j \le N, \omega_j \in \Omega, X(\omega_j) \in E_k^n \}).
$$

and switching the limits, 

$$
E(X) =  \lim_{N \rightarrow \infty} \frac{1}{N} \lim_{n \rightarrow \infty} \sum_{k=1}^n (x; x \in E_k^n)  \, \#(\{ \omega_j; 1 \le j \le N, \omega_j \in \Omega, X(\omega_j) \in E_k^n \}).
$$

$$
E(X)  =  \lim_{N \rightarrow \infty} \frac{1}{N} \lim_{n \rightarrow \infty} \sum_{k=1}^n \,\sum \{  X(\omega_j) ; 1 \le j \le N, \omega_j \in \Omega, X(\omega_j) \in E_k^n \}.
$$


$$
E(X)  =  \lim_{N \rightarrow \infty} \frac{1}{N}  \,\sum \{  X(\omega_j) ; 1 \le j \le N, \omega_j \in \Omega, X(\omega_j) \in R \}.
$$


$$
E(X)  =  \lim_{N \rightarrow \infty} \frac{1}{N}  \,\sum_{j=1}^N   X(\omega_j)
$$






\end{document}
