
\documentclass[12pt]{article}

\usepackage{amsmath}
\usepackage{amssymb}
\usepackage{amsfonts}

\usepackage{latexsym}
\usepackage{graphicx}
\usepackage{colonequals}
\usepackage{enumerate}

\setlength\topmargin{-1in}
\setlength{\oddsidemargin}{-0.5in}
%\setlength{\evensidemargin}{1.0in}

%\setlength{\parskip}{3pt plus 2pt}
%\setlength{\parindent}{30pt}
%\setlength{\marginparsep}{0.75cm}
%\setlength{\marginparwidth}{2.5cm}
%\setlength{\marginparpush}{1.0cm}
\setlength{\textwidth}{7.5in}
\setlength{\textheight}{10in}


\usepackage{listings}


\newcommand{\pset}[1]{ \mathcal{P}(#1) }
\newcommand{\partset}[1]{ \mathcal{P}^{*}(#1) }

\newcommand{\st}[0]{ \textrm{ s.t. } }
\newcommand{\fall}[0] { \textrm{ for all } }
\newcommand{\wrt}[0] { \textrm{ w.r.t. } }
\newcommand{\aew}[0] { \textrm{a.e.} }
\newcommand{\where}[0] { \textrm{ where } }


\newcommand{\IF}[0] { \; \textrm{if} \; }
\newcommand{\THEN}[0] { \; \textrm{then} \; }
\newcommand{\ELSE}[0] { \; \textrm{else} \; }
\newcommand{\AND}[0]{ \; \textrm{ and } \;  }
\newcommand{\OR}[0]{ \; \textrm{ or } \; }

\newcommand{\rimply}[0] { \Rightarrow }
\newcommand{\limply}[0] { \Leftarrow }
\newcommand{\rlimply}[0] { \Leftrightarrow }
\newcommand{\lrimply}[0] { \Leftrightarrow }

\newcommand{\rarw}[0] { \rightarrow }
\newcommand{\larw}[0] { \leftarrow }

\newcommand{ \defeq }[0] { \colonequals }
\newcommand{ \eqdef }[0] { \equalscolon }
\newcommand{ \cf }[1] { \mathbf{1}_{#1} }

\renewcommand{\Re}{ \operatorname{Re} }
\renewcommand{\Im}{ \operatorname{Im} }

\newcommand{\nats}[0] { \mathbb{N}}
\newcommand{\reals}[0] { \mathbb{R}}
\newcommand{\cmplxs}[0] { \mathbb{C}}
\newcommand{\complexes}[0] { \mathbb{C}}
\newcommand{\scalars}[0] { \mathbb{F}}
\newcommand{\exreals}[0] {  [-\infty,\infty] }

\newcommand{\eps}[0] {  \epsilon }
\newcommand{\om}[0] { \omega }
\newcommand{\Om}[0] { \Omega }

\newcommand{\A}[0] { \mathcal{A} }
\newcommand{\B}[0] { \mathcal{B} }
\newcommand{\C}[0] { \mathcal{C} }
\newcommand{\D}[0] { \mathcal{D} }
\newcommand{\E}[0] { \mathcal{E} }
\newcommand{\F}[0] { \mathcal{F} }
\newcommand{\G}[0] { \mathcal{G} }
\newcommand{\M}[0] { \mathcal{M} }
\newcommand{\cS}[0] { \mathcal{S} }
\newcommand{\U}[0] { \mathcal{U} }
\newcommand{\V}[0] { \mathcal{V} }
\newcommand{\W}[0] { \mathcal{W} }

\newcommand{\Bl}[0] { \mathcal{B} \ell }
\newcommand{\Ell}[0] { \mathcal{L} }

\newcommand{ \Ball } { \textrm{Ball} }
\newcommand{ \Var } { \textrm{Var} }
\newcommand{ \Ker } { \textrm{Ker} }
\newcommand{ \supp } { \textrm{supp} }

\newcommand{ \SUP }[1] { \sup \left( \left\{ #1 \right\} \right) }



\begin{document}

\break

\begin{flushleft}
Complex measures. \\
\end{flushleft}

\begin{flushleft}
\addvspace{5pt} \hrule
\end{flushleft}	

For $\A$ a $\sigma$-algebra, $E \in \A$, define $\partset{E, \A} \defeq \{ \{ E_k \in \A; k \in \nats \}; E = \cup_k E_k, E_i \cap E_j = \phi \;  \forall i \not = j \}$. Always $ \{ \phi, E \} \in \partset{E, \A}$, so $\partset{E, \A}$ is never empty, and also $\partset{\phi, \A} = \{ \phi \}$.    \\

For $\A$ a $\sigma$-algebra, define 


\noindent
$S^\pm(\A) \defeq \{ \sum_{k=1}^n c_k \, \cf{E_k}; \{c_k \in\reals; c_i \not = c_j \, \forall i \not = j\}, \{E_k \in \A; E_i \cap E_j = \phi \, \forall i \not = j \}, n \in \nats \}$ \\
\noindent
$S^{+}(\A) \defeq \{ \sum_{k=1}^n c_k \, \cf{E_k}; \{c_k \in [0, \infty], c_i \not = c_j \, \forall i \not = j\}, \{E_k \in \A; E_i \cap E_j = \phi \, \forall i \not = j \}, n \in \nats \}$ \\



A complex or a signed and finite measure on a measurable space $(X,\A)$ is a function, $\nu$, from $\A$ to $\reals$ or $\cmplxs$ such that

1) $\nu(\phi) = 0$

2) $\nu( \cup_{k\in \nats} E_k ) = \sum_{k \in \nats} \nu(E_k) $, $E_k \in A$, disjoint.  \\

Because the union in (2) is independent of the labeling of the $\{ E_k \}$, the sum in (2) is rearangement-invariant, which implies that it converges iff it does so to absolutely, and does to the same number. \\

Alternatively, a complex measure $\nu$ on $(X,\A)$ is a complex function on $\A$ such that

3) $\nu(E) = \sum_{k \in \nats} \nu(E_k), \fall \{ E_k\} \in \partset{E, \A}$.\\

$(3 \rimply 1), \phi = E = \cup_k E_k \rimply E_k = \phi. \rimply \nu(\phi) = \sum_{k \in \nats } \nu(\phi) \rimply \nu(\phi) = 0.$ $(3 \lrimply 2), E := \cup_k E_k.$ \\



Write $\M(X,\A)$ or $\M(\A)$ for the set of all complex or signed and finite measures on $\A$.
Write $\M^\pm(X,\A)$ or $\M^\pm(\A)$ for the set of all signed and finite measures on $\A$.
Write $\M^+(X,\A)$ or $\M^+(\A)$ for the set of all positive measures on $\A$. 
Positive measures need not be finite, so $\M^+(\A) \not \subset \M(\A)$.
\\



If $\mu_1,\mu_2 \in \M^+(\A)$, then we say that $\mu_1 \le \mu_2$ iff $\mu_1(E) \le \mu_2(E)$ for all $E \in \A$.\\

Given $\nu \in \M(\A)$, we wish to find the smallest $\mu \in \M^+(\A)$ s.t.

$$
     \mu(E)  \ge |\nu(E)| \fall E \in \A  \; {}^{(\dagger_1)},
$$
 
\noindent
smallest in the sense of the previous point. When ${(\dagger_1)}$ holds we say that $\mu$ dominates $\nu$.  Let $\{ E_k \} \in \partset{E,\A}$ arbitrarily, we then have that $|\nu(E_k)| \le \mu(E_k)$ for all $E_k$ by ${(\dagger_1)}$, summing these gives 

$$
\mu(E) = \sum_{k \in \nats} \mu(E_k) \ge \sum_{k \in \nats} |\nu(E_k)| \ge | \nu(E)|, \; \fall \{ E_k\} \in \partset{E, \A}.
$$

\noindent
Thus, for any $\mu$ dominating $\nu$, we can find a $\sum_{k \in \nats} |\nu(E_k)|$ not strictly between any $\mu(E)$ and $|\nu(E)|$. So the best we could do, in the sense of minimizing ${(\dagger_1)}$, is $\sum_{k \in \nats} |\nu(E_k)|$, for some $ \{ E_k\} \in \partset{E, \A}$ which minimizes this quantity. This suggests the definition\\

$$
|\nu|(E) := \sup \left\{  \sum_{k \in \nats} |\nu(E_k)| ; \; \{ E_k\} \in \partset{E, \A}  \right\}.
$$

\noindent   
Briefly, ${(\dagger_1)}$ holds because this $\sup$ is an upper bound, and the ``smallest'' criterion holds because the $\sup$ is the smallest such upper bound. This quantity is called the total variation measure of $\nu$. \\






(Rudin 6.2) $(X, \A)$ measurable, $\nu \in \M(X,\A)$, then $|\nu| \in \M^+(X, \A)$, and $|\nu| \le \mu$ for all $\mu  \in \M^+(X, \A)$ satisfying  $\mu(E)  \ge |\nu(E)| \fall E \in \A \; {}^{(\dagger_1)} $. \\


Proof: \\

\noindent
First, for $E \in \A$, let $ F = \{  \sum_{k \in \nats} |\nu(E_k)| ; \; \{ E_k\} \in \partset{E, \A}  \}$ is a well defined set, because $\partset{E, \A} \subset \A$, so that these sums are well defined. Note that $F \subset \reals$, if $F$ is unbounded, then $|\nu|(E) = \infty$, otherwise $F$ is bounded and this $\sup(F) \in \reals$ exists. \\


\noindent
 $\partset{\phi, \A} = \{ \phi \} \rimply |\nu|(\phi) = |\nu(\phi)| = 0$. 

\noindent 
For any $\{ E_i \} \in \partset{E, \A}$, that $|\nu|(E) = \sum_{i \in \nats} |\nu|(E_i)$ follows by ``$\le$'' and ``$\ge$'' cases. \\


\noindent


\noindent
``$\ge$'': If $|\nu|(E) = \infty$ then this case always holds, so assume $|\nu|(E) < \infty$. $\{ E_i \} \in \partset{ E, \A }$ is given. Pick $\{ t_i \in \reals; i \in \nats, t_i \ge 0 \}$ such that $t_i < |\nu|(E_i)$, but if $|\nu|(E_i) = 0$, then let $t_i = 0$. Given each $t_i$ we can find a partition of $E_i$, $\{A_{i,j} \} \in \partset{E_i, \A}$, such that $\sum_{j \in \nats} |\nu(A_{i,j})| \ge t_i$. Each $E_i$ has atleast one well defined partition; at a minimum $\{E_i, \phi \} \in \partset{E_i, \A}$. If this is the only partition in $\partset{E_i, \A}$, then $|\nu|(E_i) = |\nu(E_i)|$, and in this case $\sum_{j \in \nats} |\nu(A_{i,j})| = |\nu(E_i)| = |\nu|(E_i) > t_i$.  Now we have that $\{ A_{i,j}; i,j \in \nats \} \in \partset{E, \A}$, so that by the $\sup$ in the difinition of $|\nu|$, summing over $i$,

$$    |\nu|(E) \ge \sum_{i =1 }^\infty \sum_{j =1}^\infty |\nu(A_{i,j})| \ge \sum_{i=1}^\infty t_i. $$

\noindent
Lemma about $\reals$, Let $L \in \reals, \{ a_k \in \reals; k \in \nats \}$, then

$$
\left(  \sum_{k=1}^n t_k \le L  \fall \{ t_k \} \in \reals^n, \; t_k \le a_k, \; n \in \nats  \right) \rimply \sum_{k = 1} ^ \infty a_k \le L.
$$

\noindent
If $n=1$, suppose $a > L$, then can find some $t \in \reals \st L < t \le a$, so the statement $ \left( t \le L  \; \forall \; t \in \reals, t \le a \right)$ contradicts $a > L$, but either $a > L$ or  $a \le L$. Suppose the lemma is true for the case $n \in \nats$, fixed, then $  \sum_{k=1}^{n+1} t_k \le L  \fall \{ t_k \} \in \reals^{n+1}, \; t_k \le a_k  \rimply$
$  \sum_{k=1}^{n} t_k \le L-t_{k+1}  \fall \{ t_k \} \in \reals^{n}, t_{k+1} \in \reals, \; t_k \le a_k \rimply$ ( by statement is true for $n \in \nats$)
$  \sum_{k=1}^{n} a_k \le L-t_{k+1},  t_{k+1} \in \reals, \; t_{k+1} \le a_{k+1} \rimply$
$  t_{k+1} \le L-\sum_{k=1}^{n} a_k,  t_{k+1} \in \reals, \; t_{k+1} \le a_{k+1} \rimply$
( by statement is true for $n=1$) $a_{k+1} \le L-\sum_{k=1}^{n} a_k \rimply $  $\sum_{k=1}^{n+1} a_k \le L$. \\


\noindent
Now, using this lemma, with $L = |\nu|(E)$, $a_k = |\nu|(E_k)$, and ${t_k}$ chosen so that ${t_k} < |\nu|(E_k)$ as previously, and relying on the result that $ \sum_{k=1}^\infty t_k \le  |\nu|(E) $, we have that

$$
\sum_{k=1}^\infty |\nu|(E_k) \le |\nu|(E).
$$



\noindent
``$\le$'':  $\{ E_k \} \in \partset{ E, \A }$ is given. Then for all $\{ A_j \} \in \partset{ E, \A }$, $\{ A_j \cap E_k; k \in \nats \} \in \partset{ A_j, \A }$, then

$$
    \sum_{j \in \nats} | \nu(A_j) | = \sum_{j \in \nats} | \sum_{k \in \nats} \nu( A_j \cap E_k)| 
    \le \sum_{j \in \nats} \sum_{k \in \nats} | \nu(A_j \cap E_k) | 
    =   \sum_{k \in \nats} \sum_{j \in \nats} | \nu(A_j \cap E_k) | \le \sum_{k \in \nats} |\nu|(E_k),
$$


\noindent
by $\{ A_j \cap E_k; j \in \nats \} \in \partset{ E_k, \A }$. This was for all $\{ A_j \} \in \partset{ E, \A }$, so is true for the $\sup$ in the definition of $|\nu|$, so

$$
\sum_{k=1}^\infty |\nu|(E_k) \ge |\nu|(E).
$$

\noindent
So we have that $|\nu| \in \M^+(X, \A)$. That $|\nu|(E) \ge |\nu(E)|$ follows by noting that $ \{ E, \phi \} \in \partset{E,\A}$ so that $|\nu|(E) \ge |\nu(E)| + |\nu(\phi)| = |\nu(E)|$. Suppose $\mu \in \M^+(\A)$ was another positive measure satisfying ${(\dagger_1)}$, then for $\{ E_k \} \in \partset{E,\A}$ arbitrarily, applying ${(\dagger_1)}$ and summing, $\sum_{k \in \nats} |\nu(E_k)| \le \sum_{k \in \nats} \mu(E_k) = \mu(E)$, now by its definition, $|\nu|(E)$ is the $\sup$ of numbers of the form on the LHS, and by this inequality, $\mu(E)$ is an upper bound for such numbers, thus $|\nu|(E) \le \mu(E)$, for all $E \in \A$.  $\Box$ \\

$\M(X, \A)$ is a vector space, with respect to measure addition, $(\nu_1 + \nu_2)(E) = \nu_1(E) + \nu_2(E)$, scaling, $\lambda(\nu)(E) = (\lambda \nu)(E)$, the zero measure, $0(E) = 0$ for all $E \in \A$. The details to this are obvious.\\

$\nu_1, \nu_2 \in \M(X, \A)$, $\lambda \in \reals$ or $\complexes$ then $|\nu_1 + \nu_2| \le |\nu_1| + |\nu_2|$, $|\lambda \nu_1| = |\lambda| \, |\nu_1|$. \\

\noindent
Proof: For all $E \in \A$, $|(\nu_1 + \nu_2)(E)| = |\nu_1(E) + \nu_2(E)| \le |\nu_1(E)| + |\nu_2(E)|$ 
$\le |\nu_1|(E) + |\nu_2|(E) = (|\nu_1| + |\nu_2|)(E)$. Scaling follows by $|\lambda \nu_1(E)| = |\lambda| |\nu_1(E)|$, and for any $A \subset \reals, a \in \reals$, $A \not = \phi, a > 0$   $\sup \{ a x: x \in A \} = a \sup A$.\\


Theorem 6.4 in Rudin: If $\nu \in \M(X, \A)$, then $|\nu|(X) < \infty$. \\

$\M(X, \A)$ is a normed space w.r.t. $||\nu|| := \nu(X) \fall \nu \in \M(X, \A)$. \\

\noindent
Proof: $|| \nu_1 + \nu_2 || = |\nu_1 + \nu_2|(X) \le |\nu_1|(X) + |\nu_2|(X) = ||\nu_1|| + ||\nu_2|| $. $||\lambda \nu|| = |\lambda \nu|(X) = |\lambda| |\nu|(X) = |\lambda| ||\nu||$. $||\nu||(X) = |\nu|(X) \ge 0$, $||\nu|| = 0 \rimply |\nu|(X) = 0 \rimply 0 = |\nu|(X) \ge |\nu|(E) \ge |\nu(E)| \fall E \in \A \rimply$ $\nu = 0$. \\

$\M(X, \A)$ is a complete metric space with respect to the canonical metric induced by the norm: $d(\nu_1,\nu_2) = (\nu_1-\nu_2)(X)$. Thus $\M(X, \A)$ is a Banach space. \\

\noindent
Proof: ADD \\

Jordan decomposition: For all $ \nu \in \M^\pm(X, \A)$, , define $\nu^+ = \frac{1}{2}(|\nu| + \nu)$, $\nu^- = \frac{1}{2}(|\nu| - \nu)$. Then $\nu^+, \nu^- \in  \M^+(X, \A)$, $\nu = \nu^+ - \nu^-$, $|\nu| = \nu^+ + \nu^-$. This is the Jordan decomposition of $\nu$, and is unique. Further, if $\nu \in \M(X, \A)$, then define $\Re(\nu)(E) = \Re(\nu(E))$, $\Im(\nu)(E) = \Im(\nu(E))$ for all $E \in \A$, then $\Re(\nu), \Im(\nu) \in \M(X, \A)$, and so $\nu = \sum_{k=0}^3 i^k \nu_k$, where each $\nu_k \in \M^+(X, \A)$, $i = \sqrt{-1}$.\\

\noindent
Proof: ADD. \\

%Proof: That $\nu^\pm \in \M(X, \mu)$ follows from fact that $\M(X, \mu)$ is a Banach space (is closed under $|\cdot|$ by ( and is closed under measure addition and scaling).    \\

For all $f:X \rarw \complexes$, $\A$-measurable, $\nu \in \M(X, \A)$, say that $f$ is $\nu$-integrable if it is $|\nu|$-integrable, so $f \in \Ell (X, |\nu|)$. Write $\nu_0 = \Re(\nu)^+, \nu_1 = \Re(\nu)^-, \nu_2 = \Im(\nu)^+, \nu_3 = \Im(\nu)^-$, then \\

$$
\int_X f d\nu = \sum_{k=0}^3 \, i^k \, \int_X f \, d\nu_k
$$

\noindent
and $f \in \Ell (X, |\nu|)$ iff $|f| \in \Ell (X, |\nu|)$ iff $|f| \in \Ell (X, |\nu|)$ iff $|f| \in \Ell (X, \nu_k)$ iff $f \in \Ell (X, \nu_k)$, for all $k \in \{0,1,2,3\}$. \\

\noindent
Proof: ADD \\




%-----------------------------

\break

For $\nu \in \M(X, \A)$, say that $\nu$ is concentrated on $A \in \A$ if $\nu(E) = \nu(A \cap E)$ for all $E \in \A$. \\

\noindent
This is equivalent to $\nu(E) = 0 \fall E \in \A, E \subset A^c$, by $\nu(E) = \nu(E \cap A) + \nu( E \cap A^c) = \nu(E \cap A) \lrimply \nu( E \cap A^c) = 0$.  Not equivalently that $\nu(A^c) = 0$. \\


\noindent
If $A,B \in \A$, and $\nu$ is concentrated on both $A,B$, then $\nu(A \setminus B) = \nu( A \cap B^c) = \nu( (A \cap B^c) \cap B ) = \nu( \phi ) = 0$, thus $\nu( A \Delta B) = 0$. So sets on which a measure concentrate differ by at most sets of measure zero. \\


\noindent
If $A,T \in \A$, and $\nu$ is concentrated on $A$, $A \cap T = \phi$, then $\nu(T) = 0$, and for all $E \in \A$, $\nu(E) = \nu(E \cap A) + 0 = \nu(E \cap A) + \nu(E \cap T) = \nu(E \cap (A \cup T))$. So if $\nu$ is concentrated on $A$, and $B \in A$ is any other set which contains $A$, then $\nu$ is concentrated on $B$ also. \\


\noindent
Then, if $\nu_1, \nu_2 \in \M(\A)$, $\nu_1$ concentrated on $A_1$, $\nu_2$ on $A_2$, then $\nu_1$ and $\nu_2$ both concentrated on $A_1 \cup A_2$, so for $\nu = \nu_1 + \nu_2$, $\nu(E) = \nu_1(E) + \nu_2(E) = \nu_1(E \cap (A_1 \cup A_2) ) + \nu_2(E \cap (A_1 \cup A_2) ) = \nu(E \cap (A_1 \cup A_2))$, so $\nu$ concentrated on $A_1\cup A_2$. Clearly sets of concentration don't change when scaling a measure by non-zero scalar. \\



\noindent
This all suggests the following construction.

$$
    \bigcap \left\{ A \in \A ; \; \nu(E) = \nu(E \cap A) \fall E \in \A  \right\}
$$

\noindent
Is this set well defined? Need to show that it is in $\A$. \\


For $\nu \in \M(X, \A)$, $\mu \in \M^+(X, \A)$, say that $\nu$ is absolutely continuous w.r.t. $\mu$ if $\mu(E) = 0 \rimply \nu(E) = 0$ $\fall E \in \A$, and write $\nu \ll \mu$. \\

For $\nu_1, \nu_2 \in \M(X, \A)$, say $\nu_1$ and $\nu_2$ are mutually singular if they are concentrated on disjoint sets, and write $\nu_1 \perp \nu_2$. \\

\noindent
If $\nu \perp \nu$, then for any sets $A,B \in \A$ on which $\nu$ concentrates, $A \cap B = \phi$, but $\nu(A \Delta B) = 0$, so $\nu(A \cup B) = 0$, so $\nu$ concentrates only on $\nu$-null sets, so $\nu=0$. \\


(Rudin 6.8) For $\mu \in \M^+(X, \A)$, $\nu,\nu_1,\nu_2 \in \M(X, \A)$, \\

\noindent
a) $\nu$ concentrated on $A \rimply$ $|\nu|$ concentrated on $\A$. \\
b) $\nu_1 \perp \nu_2 \rimply |\nu_1| \perp |\nu_2|$. \\
c) $\nu_1 \perp \mu$, $\nu_2 \perp \mu \rimply \nu_1 + \nu_2 \perp \mu$. \\
d) $\nu_1 \ll \mu, \nu_2 \ll \mu \rimply \nu_1 + \nu_2 \ll \mu$. \\
e) $\nu \ll \mu \rimply |\nu| \ll \mu$. \\
f) $\nu_1 \ll \mu, \nu_2 \perp \mu \rimply \nu_1 \perp \nu_2$. \\
g) $\nu \ll \mu, \nu \perp \mu \rimply \nu = 0$. \\

\noindent
a) Let $E \subset A^c$, $\{ E_k \} \in \partset{E, \A}$, then $\nu(E_k) = 0$, $\{E_k\}$ is arbitrary, so $|\nu|(E) = 0$. \\

\noindent
b) If $\nu_1$ is concentrated on $A_1$, $\nu_2$ on $A_2$, then $|\nu_1|$ on $A_1$, and $|\nu_2|$ on $A_2$ by (a). The hypothesis is that $A_1$ and $A_2$ are disjoint, this is then what is needed for the conclusion. \\

\noindent
c) $\nu_1$ concentrated on $A_1$, $\mu$ on $B_1$, $\nu_2$ on $A_2$, and $\mu$ on $B_2$, so the hypothesis is that that $A_1 \cap B_1 = \phi$, $A_2 \cap B_2 = \phi$, but $\nu_1 + \nu_2$ is concentrated on $A_1 \cup A_2$, and $\mu$  on $B_1 \cup B_2$, and $(A_1 \cup A_2) \cap (B_1 \cup B_2) = (A_1 \cap B_1) \cup (A_2 \cap B_2) = \phi$. \\

\noindent
d) If $\mu(E) = 0$, then $\nu_1(E) = 0 = \nu_2(E)$, so $\nu_1(E) + \nu_2(E) = (\nu_1 + \nu_2)(E) = 0$.\\

\noindent
e) If $\mu(E) = 0$, then for any $\{ E_k \} \in \partset{E, \A}$, $\mu(E_k) = 0$, thus $\nu(E_k) = 0 \fall E_k$ bu hypothesis, and thus $|\nu|(E) = 0$.   \\

\noindent
f) Since $\nu_2 \perp \mu$, there is an $A \in \A$ with $\mu(A) = 0$, and $\nu_2$ concentrated on $A$. $\nu_1(E) = 0 \fall E \subset A, E \in \A$ by $\mu(E) = 0$, thus $\nu_1$ concentrates on $A^c$.  \\

\noindent
g) by (f), $\nu \perp \nu$, so $\nu = 0$.  \\


(Rudin 6.9) For $\mu \in \M^+(X, \A)$, $\sigma$-finite, then there is a function $w \in L^1(\mu) \st w(x) \in (0,1) \fall x \in X$. Thus $\tilde{\mu}(E) := \int_E w \, d\mu \in \M^+(X, \A)$, and $\mu(E) = 0 \lrimply \tilde{\mu}(E)=0 \fall E \in \A$   and $ \tilde{\mu}(X) \le 1 < \infty$. \\

\noindent
Proof: $X = \cup_k E_k, E_k \in \A, \mu(E_k) < \infty$. Let $w_k(x) = \chi_{_{E_k}}(x) \; 2^{-k}/ (1+\mu(E_k))$. Then on $E_k, 1+\mu(E_k) \in (1,\infty), 1/ (1+\mu(E_k)) \in (0,1)$. Then $w(x) = \sum_{k=1}^\infty w_k(x) \in (0,1)$. Each $w_k$ is the product of a measurable characteristic function, and a number, and so is measurable, and then by Beppo-Levi,

$$
    \tilde{\mu}(X) =  \int_X \sum_{k=1}^\infty w_k(x) d\mu(x) =  \sum_{k=1}^\infty  \int_X w_k(x) d\mu(x) =  \sum_{k=1}^\infty   \frac{2^{-k}}{ (1+\mu(E_k))}     \mu(E_k) \le \sum_{k=1}^\infty 2^{-k} = 1.
$$

$$
    \tilde{\mu}(E) =  \sum_{k=1}^\infty  \int_X \chi_{_{E}}(x) \frac{ \chi_{_{E_k}}(x)  2^{-k}}{ (1+\mu(E_k))} d\mu(x) = \sum_{k=1}^\infty \frac{ \mu(E \cap E_k ) }{ 1+\mu(E_k)} 2^{-k} = 0 \lrimply \mu(E) = 0
$$

\noindent
because $\mu(E) = \sum_{k=1}^\infty \mu(E \cap E_k)$.


\vspace{20pt}

(Rudin 1.40) $\mu \in \M^+(\A)$, $\mu(X) < \infty$, $f: X \rarw \scalars$, $f \in L^1(\mu)$, $S \subset \scalars$, closed, and define the averages

$$
A_E(f) = \frac{1}{\mu(E)}  \int_E f \, d\mu.
$$

\noindent
If $A_E(f) \in S$ for all $E \in \A$ with $\mu(E) > 0$, then $f(x) \in S$ for $\mu$-a.e. $x \in X$. \\

\noindent
Proof: Let $r>0$ and $z \in \scalars$ such that $B_{z,r} \defeq \{ y \in \scalars; |z-y| \le r \} \subset S^c$. Let $E = f^{-1}(B_{z,r})$, suppose that $\mu(E) > 0$. Then $|A_E(f) - z| = |A_E(f) - \frac{\mu(E)}{\mu(E)} z | = | \frac{1}{\mu(E)} \int_E f \, d\mu - \frac{1}{\mu(E)} z \int_E 1 \, d\mu | = |\frac{1}{\mu(E)} \int_E (f(x) - z) \, d\mu(x)| \le \frac{1}{\mu(E)} \int_E |f(x) - z| \, d\mu(x) |\frac{1}{\mu(E)} \int_E (f(x) - z) \, d\mu(x)| \le \frac{1}{\mu(E)} \int_{ \{ x \in X; |f(x)-z| \le r \} } |f(x) - z| \, d\mu(x) \le   \frac{1}{\mu(E)} \int_E r \, d\mu(x)  = r$. So $\mu(E) > 0 \rimply |A_E(f) - z| \le r \rimply A_E(f) \in B_{z,r} \subset S^c$, but the hypothesis is that $A_E(f) \in S$ for all $E \in \A$, thus $\mu(E) = 0$. Since any open set in $\scalars$ is a union of open balls, and hence a countable union of closed balls like $B_{z,r}$, $\mu(f^{-1}(S^c)) = \mu(  f^{-1} ( \cup_{k \in \nats} B_k ) ) \le  \mu(   \sum_{k \in \nats} f^{-1} ( B_k ) )  = 0$. \\


If $(X, \A, \mu)$ a (positive) measure space, $f,g:X \rarw \scalars \in L^1(\mu)$. If $\int_E f \, d\mu = \int_E g \, d\mu \fall E \in \A$, then $f=g$ $\mu-$a.e., this is also true for measurable $f,g:X \rarw [0, \infty]$ if $\mu$ is $\sigma$-finite. \\

\noindent 
Proof: $f,g$ integrable implies $\int_E f \, d\mu - \int_E g \, d\mu  = 0 \rimply \int_E (f - g) \, d\mu  = 0$. Let $h = f-g$, then let $A_E = \frac{1}{\mu(E)} \int_E h \, d\mu$, so $A_E = 0$ for all $E \in \A$, and $\{ 0 \}$  is a closed subset of $\scalars$ (finish) \\

\noindent
%For measurable $f,g:X \rarw [0, \infty]$, and if $X = \cup_{k \in \nats} X_k$, $\mu(X_k) < \infty$, re-apply to each $X_k$ individually with $f_k = f|_{X_k}, g_k = g|_{X_k}$ and $\A_k = A_{X_k}$. Thus we may assume that $\mu(X) < \infty$. \\
For measurable $f,g:X \rarw [0, \infty]$, and if $X = \cup_{k \in \nats} X_k$, $\mu(X_k) < \infty$. Fix $U = E_k$, some $k$, so $\mu(U) < \infty$, let $F = U \cap f^{-1}(\{\infty\})$, then let $F'_n = F \cap g^{-1}([0,n))$. Suppose $F'_n \not = \phi$, then $\int_{F'_n} f \, d\mu = \infty$ because $f = \infty$ everywhere on $F'_n$, but $\int_{F'_n} g \, d\mu \le n \mu(F') < \infty$, but $\int_{F'_n} f \, d\mu = \int_{F'_n} g \, d\mu$, so $F'_n= \phi$ for all $n \in \nats$. Now $(g^{-1}(\{\infty\}))^c = \cup_{n=1}^\infty g^{-1}([0, n))$, $F \cap (g^{-1}(\{\infty\}))^c = \cup_{n=1}^\infty F \cap g^{-1}([0, n)) = \cup_{n=1}^\infty \phi = \phi$, but 
$F = F \cap (g^{-1}(\{\infty\}))^c \cup F \cap (g^{-1}(\{\infty\})) = F \cap (g^{-1}(\{\infty\}))$ (finish) \\



% \cap_{m=1}^\infty \cup_{n=m}^\infty g^{-1}([0, n))

If $(X, \A, \mu)$ a measure space, $[f] \in L^1(\A,\mu)$, then $\nu \in \M(\A)$ and $\nu \ll \mu$, where

$$ \nu(E) = \int_E f \, d\mu \; \fall E \in \A.$$

\noindent
Proof: First, $|f \cf{E}| \le |f| \rimply f \cf{E} \in  L^1(\A,\mu)$ so $|\nu(E)| < \infty \fall E \in \A$, if $w \in [f]$ arbitrarily, then $\int_E f \, d\mu = \int_E w \, d\mu  \; \fall E \in \A $, because $f = w$ $\mu$-a.e., so $\nu$ is independent of the choice of functions in $[f]$, and so $\nu$ is finite and well defined. $\nu(\phi) = \int_X \cf{\phi} f \, d\nu = \int_X 0 \, d\nu = 0$. If $\{ E_k \}_{k \in \nats} \in \A$, disjoint, $E  = \cup_{k \in \nats} E_k$, then $\nu(E) = \int_X \cf{E} f \, d\mu = \int_X ( \sum_{k \in \nats} \cf{E_k} ) f \, d\mu$. Now, let $h_n(x) = \sum_{k=1}^n \cf{E_k}(x) f(x)$, then each $h_n$ is $\A$-measurable, $\lim_{n \rarw \infty} h_n(x) f(x) = \cf{E}(x) f(x)$, which is $\A$-measurable, and $|h_n(x)| \le |f(x)|$, and $|f| \in L^1(\A, \mu)$. So by LDCT, $\nu(E) = \int_X ( \sum_{k \in \nats} \cf{E_k} ) f \, d\mu  =  \sum_{k \in \nats} \int_{E_k}  f \, d\mu = \sum_{k \in \nats} \nu(E_k)$, so $\nu \in \M(\A)$. Next, $f(x) \le \sup \{ f(x); x \in E \} \fall x \in E \rimply \nu(E) = \int_E f \, d\mu \le \sup \{ f(x); x \in E \} \int_E \, d\mu = \sup \{ f(x); x \in E \} \mu(E)$, so $\mu(E) = 0 \rimply \nu(E) = 0$, so $\nu \ll \mu$.







%---------------


\break

Rudin 6.10: $(X, \A)$ a measurable space, $\mu \in \M^+(X, \A)$, $\sigma$-finite. \\

\noindent 
a) Lebesgue Decomposition Theorem (LDT): \\

\noindent 
For all $\nu \in \M(X, \A)$, there exist unique $\nu_a, \nu_s \in \M(X, \A)$ such that $\nu = \nu_a + \nu_s$, $\nu_a \ll \mu$, $\nu_s \perp \mu$, $\nu_a \perp \nu_s$. \\

\noindent 
b) Radon-Nikodym Theorem (RNT): \\

\noindent 
For all $\nu \in \M(X, \A)$ such that $\nu \ll \mu$, there exists a unique $[h] \in L^1(\mu)$ such that $ \nu(E) = \int_E h \, d\mu $, $E \in \A$. \\

\vspace{10pt}

\noindent
Remarks:  \\

\noindent
In the LDT, we can apply the RNT to $\nu_s$, and if $\nu \ll \mu$, then $\nu_s = 0$. \\

\noindent
In the LDT and RNT, $\nu \ge 0$, then $h, \nu_a, \nu_s \ge 0$ $\mu$-a.e. \\

\noindent
In the LDT and RNT, if $\nu \in \M^+(\A)$, then these results hold with $\nu_a, \nu_s \in \M^+(X, \A)$, but  $h: X \rarw [0, \infty]$, $\A$-measurable, might not be integrable, but will be the (infinite) sum of locally integrable functions. \\

\noindent
Notation: \\


\noindent
Given $\nu \in \M(X, \A)$, write $\frac{d\nu}{d\mu}$ for the $\mu$-integrable function such that $\nu_a(E) = \int_E \frac{d\nu}{d\mu} \, d\mu$, for all $E \in \A$, where $\nu_a$ is the part in the LDT such that $\nu_a \ll \nu$, then $d\nu =\frac{d\nu}{d\mu} d\mu$. $\frac{d\nu}{d\mu}$ is called the Radon-Nikodym derivative of $\nu$ with respect to $\mu$. More generally, take $d\nu = w \, d\mu$ to mean $\nu(E) = \int_E w \, d\mu$ for all $E \in \A$. \\


%\noindent
%If, $\nu \in \M^+(X, \A)$ but not $\sigma$-finite, then LDT holds, but $\nu_a, \nu_s$ might not be finite. \\

%\noindent
%If, $\nu \in \M^+(X, \A)$ but not $\sigma$-finite, then RNT holds but $h$ might not be finite or integrable. \\




\vspace{10pt}


\noindent
Proof: \\

\noindent
Uniqueness: In LDT, if $(\nu_a', \nu_s')$ another pair of measures from LDT, then $\nu_a' - \nu_a = \nu_s - \nu_s'$, $\nu_a' - \nu_a \ll \mu$, and $\nu_s - \nu_s' \perp \mu$, hence both sides here are 0, by c,d,g from preliminary propositions. In RNT, if $ \nu(E) = \int_E h \, d\mu =  \int_E h' \, d\mu $, then $  \int_E (h - h') \, d\mu = 0$, $E \in \A$, and then by the vanishing principle, $h = h'$ $\mu$-a.e.. \\

\noindent
Step 1: If $\nu \in \M^+(X, \A)$,  $\nu(X) < \infty$, then apply Rudin 6.9 to $\mu$ to obtain $w \in L^1(\mu), w(x) \in (0,1) \fall x \in X$. Then $\varphi(E) := \nu(E) + \int_E w \, d\mu$ is a positive finite measure on $\A$, and $\varphi \ge \nu$. Then for any $\A$-measurable function $f:X \rarw [0, \infty]$,

$$
\int_X f \, d\varphi = \int_x f \, d\nu + \int_x f w \, d\mu,
$$

\noindent
by following the standrad steps in the construction of the integral. If $f \in L^2(\mu)$,


$$
\int_X |f| \, d\nu  \le  \int_X |f| \, d\varphi \le \left( \int_X 1 \, d\varphi \right)^{1/2} \left( \int_X |f|^2 \, d\varphi \right)^{1/2}   = \varphi(X)^{1/2} \left( \int_X |f|^2 \, d\varphi \right)^{1/2} < \infty
$$

\noindent
by the Schwarz inequailty, so $f \in L^1(\nu), f \in L^1(\varphi)$, similarly, $fw \in L^1(\mu)$. Thus, $f \mapsto \int_X f \, d\nu$ is a linear functional, bounded (by $\sqrt{\varphi(X)}$)  on $L^2(\varphi)$. Hence by $L^2(\varphi)$ being a Hilbert space, and Riesz representation, there exists a $g \in L^2(\varphi)$ so that

$$
\int_x f \, d\nu = \int_x fg \, d\varphi, \fall f \in L^2(\varphi).
$$


\noindent
Then, for $f = \chi_{_E}$, for any $E \in \A$ with $\varphi(E) > 0$, $\lambda(E) = \int_E g \, d\varphi$, and because $ 0 \le \lambda \le \varphi$, $ 0 \le \lambda(E) / \varphi(E) \le \varphi(E)  / \varphi(E) = 1$,

$$
0 \le \frac{1}{\varphi(E)} \int_E g \, d\varphi \le 1, \fall E \in \A.
$$

\noindent
So by Rudin 1.40, $g \in [0,1]$ $\varphi-$a.e. so wlog, $g(x) \in [0,1] \fall x \in X$. \\

\noindent
Define $A = g^{-1} ([0,1))$, $B = g^{-1}( \{ 1 \})$, then $A,B \in \A$ by $g \in L^2(\varphi) \rimply$ $g$ is $\A$-measurable, $A \cup B = X$, $A \cap B = \phi$. Define $\nu_a(E) = \nu(A \cap E)$, $\nu_s(E) = \nu(B \cap E)$ for all $E \in \A$. Notice $\nu_a(E) + \nu_s(E) = \nu(E \cap A) + \nu(E \cap B) = \nu(E \cap X) = \nu(E)$ for all $E \in \A$, so $\nu = \nu_a +\nu_s$, and by definition, $\nu_s$ is concentrated on $A$, $\nu_s$ on $B$ so $\nu_a \perp \nu_s$. \\


\noindent
Now, rewriting,

$$
\int_X f \, d\nu = \int_X fg \, d\varphi =  \int_X  fg \, d\nu  +  \int_X  fgw \, d\mu \rarw 
$$
$$
 \int_X (1-g) f \, d\nu = \int_X fgw \, d\mu.
$$

\noindent

Let $f = \chi_{_B}$, then the LHS is $0$, and the RHS is $\int_X w \, d\mu$, and since $w > 0$, $\mu(B) = 0$, so $\mu$ is concentrated on $B^c = A$, so that $\nu_s \perp \mu$. Next, let $ f = \chi_{_E}  \sum_{k=0}^n g^k$, then $f \ge 0$, $f \in L^2(\varphi)$. Then,

$$
    \int_E (1-g) f \, d\nu = \int_E fgw \, d\mu \rarw \hspace{10pt} \int_E (1-g^{n+1}) \, d\nu = \int_E \sum_{k=0}^n g^{k+1} w \, d\mu.
$$

\noindent Let 

$$
h(x) = w(x)  g(x)\sum_{k=0}^\infty g^{k}(x).
$$

\noindent 
For $x \in A$, $g^k(x)$ decreases monotonically, so the partial sums in $h$ increase monotonically, and $ h(x) = \frac{g(x) w(x)}{1-g(x)}$. So, taking the limit of the equation, gives by LMCT 

$$
\lim_{n \rarw \infty}  \int_{E \cap A} (1-g^{n+1}) \, d\nu =  \lim_{n \rarw \infty}  \int_{E \cap A} \sum_{k=0}^n g^{k+1} w \, d\mu =
$$

$$
\int_{E \cap A}  \lim_{n \rarw \infty}  (1-g^{n+1}) \, d\nu =  \int_{E \cap A} \lim_{n \rarw \infty}   \sum_{k=0}^n g^{k+1} w \, d\mu =
$$

$$
\int_{E \cap A} 1 \, d\nu =  \int_{E \cap A} h \, d\mu = \nu(E \cap A) = \nu_a(E).
$$


\noindent 
For $x \in B$, $g(x) = g^k(x) = 1$, so $1-g^{n+1}(x) = 0$, so

$$
\int_{E \cap B} h \, d\nu = \int_{E \cap B} 0 \, d\nu = 0.
$$

\noindent
not finished.




\break

Folland 1.29: $(X ,\A, \mu)$ a $\sigma$-finite measure space, for $w: X \rarw [0,\infty]$, $\A$-measurable and $\nu$ as $d\nu = w \, d\mu $, then $\nu \in \M^+(\A)$, and for all $f:X \rarw [0, \infty]$ $\A$-measurable, $\int_X f\, d\nu = \int_X fw \, d\mu$. \\

\noindent
Proof: let $\{E_k\} \in \partset{E, \A}$. Notice that $\cf{E} w = \sum_{kj=1}^\infty \cf{E_k} w$, by Beppi-Levi, $\nu(E) = \sum_{k=1}^\infty \nu(E_k)$, and $\nu(\phi) = \int_X \cf{\phi} w \, d\mu = 0$, so $\nu \in \M^+(\A)$. Then letting $f = \cf{E}, E \in \A$ the formula  holds, thus for simple functions, and thus for general $f:X \rarw [0, \infty]$ using LMCT.\\



Folland 6.13: for $\nu \in \M(\A)$, $\mu \in \M^+(\A)$ $\sigma$-finite, and $d\nu = w \, d\mu$, then $d|\nu| = |w| \, d\mu$, or in other words, $\frac{d|\nu|}{\\d\mu} = |\frac{d\nu}{d\mu}|$. \\

\noindent
Proof: Need to show that $|\nu|(E) = \int_E |w| \, d\mu$ for all $E \in \A$. Notice $|\nu(E)| = |\int_E w \, d\mu| \le \int_E |w| \, d\mu$. So by definition, $|\nu| \le |\lambda|$, with $\lambda$ defined by $d\lambda = |w| \, d\mu$. For the reverse inequality, let $A = \{x \in X; w(x) \not = 0 \}$, then $A \in \A$ because $[w] \in L^1(\mu)$ so $w$ is $\A$-measurable. Define $K(x) = |w(x)| \div w(x),$ for $x \in \A$, and $K(x) = 0$ else, then $K$ is measruable, beasue $ |w(x)| \div w(x) $ is measurable $\wrt \A_A$, . Then $|K| \le 1$ and so there exists a sequence of simple functions $s_n \nearrow K$ pointwise, $|s_n| \le |K| \le 1$. Then $s_n(x) w(x) \rarw K(x) w(x) = |w(x)|$. By LDCT, $\int_E s_n h \, d\mu \rarw \int_E |w| \, d\mu$. Suppose $s_n = \sum_{k=1}^m c_k \cf{E_k}$ in its standard representation, then $|c_k| \le 1$ and 

$$
|\int_E s_n w \, d\mu | = |\sum_{k=1}^m c_k \int_{E \cap E_k} w \, d\mu| \le \sum_{k=1}^m |c_k| |\nu(E \cap E_k)|  \le \sum_{k=1}^m 1 \cdot| |\nu(E \cap E_k)| = |\nu|(E \cap (\cup_k E_k)) \le |\nu|(E),
$$

\noindent
so $\int_E |w| \, d\mu \le |\nu|(E)$. $\Box$\\

For $\nu \in \M(\A)$, $\mu \in \M^+(\A)$ $\sigma$-finite, and $d\nu = w \, d\mu$, then $d\nu_k = w_k \, d\mu_k$, where $\nu = \sum_{k=0}^3 \, i^k \, \nu_k$, $w = \sum_{k=0}^3 \, i^k \, w_k$, and each $\nu_k , w_k \ge 0$. \\

\noindent
$f^\pm = \frac{1}{2} f \pm  \frac{1}{2} f$, $\Re{f} = \frac{1}{2} f + \frac{1}{2} \overline{f}$, $\Im{f} = \frac{1}{2} f - \frac{1}{2} \overline{f}$. $f_0 = \Re{f}^+$, $f_1 = \Im{f}^+$, $f_2 = \Re{f}^-$, $f_3 = \Im{f}^-$. \\

\noindent
Proof: The Jordan decomposition of $\nu$ and the decomposition of $w$ into positive, negative, real and imaginary parts take the same form, then use the  previous theorem and the linearity of the integral. \\

For $\nu \in \M(\A)$, $\mu \in \M^+(\A)$ $\sigma$-finite, then $d\nu = w \, d\mu$ $\lrimply$ $\int_x f \, d\nu = \int_X fw \, d\mu \fall f \in L^1(X, |\nu|)$.\\

\noindent
Proof: ($\limply$) For $E \in \A$, $f = \cf{E}$, then $\int_E 1 d\nu = \int_E w \, d\mu = \nu(E)$. ($\rimply$) follows by decomposition into parts and linearity and defninition of the integral.   \\

Folland 6.12: for $\nu \in \M(\A)$ there is a $[w] \in L^1(\A, |\nu|)$ such that $d\nu = w \, d|\nu|$, with $|w| = 1$. \\

\noindent
Proof: ADD \\

Using these results, for any $\nu \in \M(\A)$, and $w$ as $d\nu = w \, d|\nu|$, then for all $f \in L^1(\A, |\nu|)$, $\int_X f \, d\nu = \int_X f w \, d|\nu|$. This may be taking as the definition of $\int_X f \, d\nu$, by integrating $f w$ in the usual way, i.e. $|fw| = |f|$ so $fw \in L^1(\A, |\nu|)$, then apply the usual steps. \\


Hahn decomposition: If $\nu \in \M^\pm(X, \A)$, then there exist disjoint $E^+, E^- \in \A, X = E^+ \cup E^-$, so that $\nu^+$ is concentrated on $E^+$ and $\nu^-$ on $E^-$, thus $\nu^+ \perp \nu^-$. \\

Proof: There exists $w: X \rarw \{-1, 1 \}$ such that $\nu(E) = \int_E w \, d|\nu|$ for all $E \in \A$; let $E^\pm = w^{-1}(\{ \pm \})$, then $E^+ \cap E^- = 0$ and $E^+ \cup E^- = X$ by properties of inverse images. Then $\nu(A \cap E^\pm) = \int_{A \cap E^\pm} w \, d|\nu| = $




\break

Duality o $L^p$ spaces: \\

\noindent
Let $1 \le p < \infty$ and let $q$ be the H\"{o}lder conjugate index of $p$. So, $\frac{1}{p} + \frac{1}{q} = 1$, $q+p = pq$, $q = pq-p$, $p=q/(q-1) = \frac{1}{1-{\frac{1}{p}}}$, and all these are hold when switching $p$ and $q$.   \\

\noindent
There is a canonical map:

$$
\Phi: L^q(X, \mu) \rarw  (L^p(X, \mu))^*, \; \Phi \defeq g \mapsto  \left( f \mapsto \int_X fg \, d\mu  \right) \; \fall g \in L^q(X, \mu), f \in L^p(X, \mu).
$$


\noindent
$\Phi$ is linear follows by linearity fo the intregral, and by H\"{o}lder's inequality, $|\Phi(g)(f) | = |\int_X fg \, d\mu| \le \int_X |fg| \, d\mu \le ||f||_p ||g||_q \le \infty$, so that $||\Phi(g)|| \le ||g||_q \fall g \in L^q(X, \mu)$ so $\Phi$ is a bounded linear functional on $L^q(X, \mu)$. \\


If $1 < p < \infty$ then this $\Phi$ is an isometry, i.e. $||\Phi(g)|| = ||g||_q \fall g \in L^q(X, \mu)$. If $p=1$ then $\Phi$ is an isometry if $X$ is $\sigma$-finite. \\

\noindent
Proof: for the first case, $p > 1$, given $g \in L^q$ let $f = \overline{\textrm{sgn}(g)} |g|^{q-1}$, so $ f(x) = \overline{g(x)} |g(x)|^{q-2}$ if $x \not = 0$, $f(x) = 0$ otherwise, $f$ is measure by usual tricks. Then $|f(x)|^p = |g(x)|^p |g(x)|^{pq-2p} = |g(x)|^{pq-p} = |g|^q$, so $g \in L^q(x, \mu) \rimply \int_X |g|^q \, d\mu  < \infty \rimply \int_X |f|^p \, d\mu < \infty \rimply f \in L^p(X, \mu)$. 












\break


$\nu \in \M^\pm(X, \A)$, then by the Hahn decomposition gives $A_+, A_- \in \A$, $A_+ \cap A_- = \phi, A_+ \cup A_- = X$, $\nu_\pm(E) = \frac{1}{2} \left( |\nu|(E) \pm \nu(E) \right) =  \pm \nu( E \cap A_\pm)$, for all $E \in \A$, and $\nu_\pm$, and then $\nu_+ \perp \nu_-$. Then $\nu_\pm$ are unique positive finite measures on $\A$. \\

\noindent
Proof: $\nu_\pm$ are signed measures, need to show that they are positive. Write $w = \frac{d\nu}{d|\nu|}$, then $|w| = 1$, $w:X \rarw \{-1,1\}$, and $A_\pm = w^{-1}(\{\pm 1\})$, then 
$\nu_\pm(E) = \frac{|\nu|(E) \pm \nu(E)}{2} = \int_E \frac{|h| \pm h}{2} \, d|\nu| = \int_E h^\pm \, d|\nu| \ge 0$ for all $E \in \A$ (integral of a positive functions wrt a positive measure). Uniqueness follows by supposing that $\tilde{A}_+, \tilde{A}_- \in \A$ another such pair, then $\nu(A_+ \cap \tilde{A}_-) = \nu_+(\tilde{A}_-)$, but $\nu_+$ is concentrated on $\tilde{A}_+$ so $\nu(A_+ \cap \tilde{A}_-) = 0$, similarly, $\nu(A_- \cap \tilde{A}_+) = 0$. \\

%$\nu \in \M^\pm(\A)$, then let $\nu_+ = \frac{1}{2}( |\nu| + \nu ), \nu_- = \frac{1}{2}( |\nu| - \nu )$. $\nu_\pm(\phi) = \frac{1}{2}( |\nu|(\phi) \pm \nu(\phi) )= 0 \pm 0 = 0$. For any $E \in \A$, $\{ E_k \} \in \partset{E, \A}$, then   $\nu(E)_\pm = \frac{1}{2}( |\nu|(E) \pm \nu(E) )= \frac{1}{2}( |\nu|(\cup_k E_k) \pm \nu(\cup_k E_k) )$ = $\cup_k \frac{1}{2}|\nu|(E_k) \pm \cup_k \frac{1}{2}\nu(E_k)$ = $\cup_k \left( \frac{1}{2}|\nu|(E_k) \pm \frac{1}{2}\nu(E_k) \right)$ = $\cup_{k \in \nats} \nu(E_k)$, so $\nu_\pm \in \M^\pm(\A)$.\\





Counter example to show that in the RNT, the positive measure needs to be $\sigma$-finite. Let $\mu$ be the counting measure on $(\reals, \pset{\reals})$, then $L^1(\reals, \mu) = \ell^1(\reals)$, and for any $\nu \in \M(\reals)$, $E$ $\nu$-measurable, $\mu(E) = 0 \rimply \sum_{x \in E} \, 1 = 0 \rimply E = \phi \rimply \nu(E) = 0$, so $\nu \ll \mu$. Then, let $(\reals, \overline{\mathcal{L}}, \overline{\lambda})$ be the complete Lebesgue measure space on $\reals$. For all $h \in \ell^1(\reals)$, let $\nu_h(E) = \int_E h \, d\nu$. Now we've seen that $h(x) = 0$ for all but countably many $x \in \reals$, which means that $\nu_h$ is concentrated on a countable set in $\reals$, $A_h$, and $A_h \in \overline{\mathcal{L}}$ by completeness, with $\overline{\lambda}(A_h) = 0$. Finally, the RNT would say that for any $\gamma \in \M(\reals, \overline{\mathcal{L}})$, there exists an $h \in \ell^1(\reals)$ such that $\gamma(E) = \nu_h(E)$ for all $E \in \overline{\mathcal{L}}$, but then there exists an $A \subset \reals$ such that $\gamma$ is concentrated on $A$; choose $\gamma(E) = \int_E w \, d\overline{\lambda}$ for all $E \in \overline{\mathcal{L}}$, some $[w] \in L^1(\overline{\lambda})$, so $\gamma$ is concetrated on a $\overline{\lambda}$-null set, because that set is countable, and $\gamma \ll \overline{\lambda}$, so $\gamma$ is concentrated on a $\gamma$-null set, so $\gamma$=0, a contratdiction for any $[w] \not = 0$. \\





Let $\M_{\mu-a.c.} = \{ \nu \in \M(\A); \nu \ll \mu \}$, $\M_{\mu-a.c.}$ is a subspace of the Banach space $\M(\A)$.\\

\noindent Let $0(E) = 0 \fall E \in \A$, then $0 \ll \mu$ trivially, so $ 0 \in \M_{\mu-a.c.}$. Let $\nu, \sigma \in \M_{\mu-a.c.}$, $c \in \mathbb{F}$, then $\lambda \defeq \nu + c \cdot \sigma \in \M(\A)$, and suppose $E \in \A$, $\mu(E) = 0$, then $\nu(E) = 0 = \sigma(0) = c \cdot \sigma(0)$, so $\lambda(E) = 0$ and so $\lambda \in \M_{\mu-a.c.}$. $\M_{\mu-a.c.}$ inherits its norm from $\M(\A)$. Suppose $(\nu_k)_{k \in \nats}$ is sequence in $\M_{\mu-a.c.}$, then $\nu = \lim_{k \rarw \infty} \nu_k \in \M(\A)$, because $\M(\A)$ is a Banach space. Suppose $E \in \A$, $\mu(E) = 0$, then each $\nu_k \in \M_{\mu-a.c.} \rimply \nu_k(0) = 0$, and then $\nu(E) = \lim 0 = 0$, so $\nu \in \M_{\mu-a.c.}$, and so $\M_{\mu-a.c.} $ is a complete metric space, and thus a Banach subspace.\\



$(X, A, \mu)$ a measure space, then the map $\Phi: L^1(\A, \mu) \rarw \M_{\mu-a.c.}$ defined by $\Phi([w])(E) = \int_E w\, d\mu$, is linear, bijective, and $|| \Phi([w]) || = ||[w]||$. So $\M_{\mu-a.c.}$ and $L^1(\A, \mu)$ are isometrically isomorphic. The linearity of $\Phi$ does not depend of $\mu$ being $\sigma$-finite.\\

\noindent
Proof: First, for any $[w] \in L^1(\A, \mu)$, then for any $E \in \A$, let $\nu(E) = \int_E w \, d\mu$. We have already shown that $\nu$ is well defined, and $\nu \in \M(\A)$, so $\Phi$ is well defined. \\

\noindent
Lemma: For $(X, \A, \mu)$ a measure space, $f,g: X \rarw \reals$, $\A$-measurable. Then $ \int_A f \,d\mu = \int_A g \, d\mu$ for all $A \in \A \rimply f = g$, $\mu-$a.e. Proof: let $h = f - g$, $A^+ = h^{-1}( [0,\infty] )$, $A^- = h^{-1}( [-\infty,0) )$. Then $h \ge 0$ on $A^+$ so $\int_{A^+} h \,d\mu = 0$ $\rimply h=0$ $\mu$-a.e. on $A^+$ by the vanishing principle. Similarly, $h^- = 0$ $\mu$-a.e. on $A^-$. \\

\noindent
Next, $\Phi([f]) = \Phi([g]) \rimply \int_E f \, d\mu = \int_E g \, d\mu$ for all $E \in \A \rimply f=g \; \;\mu-\aew \rimply f,g \in [g] = [f]$, so $\Phi$ is injective, using the lemma. If $\nu \in \M_{\mu-a.c.}$, then the RNT says that there is a unique $[h] \in L^1(\A, \mu)  \st \nu(E) = \int_E h \, d\mu  \fall E \in \A $, hence $\Phi$ is surjective. \\

\noindent
Linearity:$\fall f,g \in L^1(\A, \mu), c \in \mathbb{F}$,  $\Phi([f] + c \cdot [g])(E) = \int_E \left( f + c \cdot g \right) \, d\mu = \int_E f d\mu + c \cdot \int_E g  \, d\mu \; \; \forall E \in \A$, and 
 $\Phi([f]) + c \cdot \Phi([g]) = \int_E f d\mu + c \cdot \int_E g  \, d\mu \; \; \forall E \in \A$, so 
 $ \Phi([f] + c \cdot [g])(E) = \Phi([f]) + c \cdot \Phi([g])$. \\
 
\noindent
Lastly, let $ \nu = \Phi([f])$, then $||\Phi([f])|| = |\nu|(X) = \int_X |f| \, d\nu = ||f||_1$, by Rudin 6.13. $\Box$ \\






$(X, A)$ a measurable space, $\nu,\sigma,\mu \in \M(\A)^+$, $\sigma$-finite.

\begin{enumerate}
\item
$\nu \ll \mu, f: X \rarw [0,\infty]$ $\A$-measurable, then $\int_X f \, d\nu = \int_X f \frac{d\nu}{d\mu} \, d\mu$.
\item
$\nu \ll \mu, \sigma \ll \mu$ then $\frac{d}{d\mu}( \nu + \sigma ) = \frac{d\nu}{d\mu} +  \frac{d\nu}{d\mu}$, $\mu$-a.e.
\item
$\nu \ll \sigma \ll \mu$ then $\frac{d\nu}{d\sigma}\frac{d\sigma}{d\mu} = \frac{d\nu}{d\mu}$ $\mu$-a.e.
\item
$\nu \ll \mu$, $\mu \ll \nu$ then $\frac{d\nu}{d\mu} = \left( \frac{d\mu}{d\nu} \right)^{-1} $
\end{enumerate}

\noindent
Proof: 1) The RNT for $\M^+$ says that $\frac{d\nu}{d\mu}$ exists and $\frac{d\nu}{d\mu}: X \rarw [0,\infty]$, is $\A$-measurable, but might not be integrable wrt $\mu$, then Folland 1.29 applies to give the result. \\

% For  $f = \sum_{k=1}^n c_k \, \cf{E_k} \in S^+(\A)$, $\int_X f \,d\nu = \int_X  \sum_{k=1}^n c_k \, \cf{E_k} \,d\nu =  \sum_{k=1}^n c_k \, \nu(E_k) = \sum_{k=1}^n c_k \, \int_{E_k} \frac{d\nu}{d\mu} \,d\mu =  \int_{X} \sum_{k=1}^n c_k \, \cf{E_k} \frac{d\nu}{d\mu} \,d\mu =  \int_X f \frac{d\nu}{d\mu} \, d\mu$, by linearity of the intergral and by the RNT. For $f:X \rarw [0,\infty]$ $\A$-measurable, let $\{s_k\} \subset S^+(\A)$, $s_1 \le s_2 \le ... \le f$, $\lim_n s_n = f$. By LMCT, $ \int_X f \, d\nu = \lim_n \int_X  s_n \, d\nu =  \lim_n \int_X  s_n \frac{d\nu}{d\mu} \, d\mu = \int_X  \lim_n  \left( s_n \frac{d\nu}{d\mu} \right) \, d\mu = \int_X \left( \lim_n s_n  \right) \frac{d\nu}{d\mu} \, d\mu = \int_X f \frac{d\nu}{d\mu} \, d\mu$, again by LMCT, and by the linearity of $\lim$, i.e. for all $x \in X$, $\frac{d\nu}{d\mu}(x)$ is just a positive real number, so the conditions for LMCT carry over from $\{ s_n(x) \}$ to $\{ s_n(x) \frac{d\nu}{d\mu}(x) \}$. If $\mu$ is the zero measure then so is $\nu$, and then the above integrals are 0 for all $f$.   \\

\noindent
Proof: 2) Let $m = \nu + \sigma$, we showed that $\M_{\mu-a.c.}$ is a linear subspace of $\M(\A)$, so $m \ll \mu$. Then the RNT for $\M^+$ says that $\frac{dm}{d\mu}$ exists and $\frac{dm}{d\mu}: X \rarw [0,\infty]$, is $\A$-measurable, but might not be integrable wrt $\mu$, similarly for $\frac{d\nu}{d\mu}$ and $\frac{d\sigma}{d\mu}$, and by definition $\nu(E) + \sigma(E) = m(E) = \int_E \frac{dm}{d\mu} d\mu = \int_E \frac{d\nu}{d\mu} d\mu + \int_E \frac{d\sigma}{d\mu} d\mu = \int_E \frac{d\nu}{d\mu} + \frac{d\sigma}{d\mu} d\mu \fall E \in \A$, and as usual, $\int_E \frac{dm}{d\mu} \, d\,u = \int_E \frac{d\nu}{d\mu} + \frac{d\sigma}{d\mu} d\mu \fall E \in \A \rimply \frac{dm}{d\mu} = \frac{d\nu}{d\mu} + \frac{d\sigma}{d\mu} \; \mu - \aew$.  \\


\noindent
Proof: 3)  he RNT for $\M^+$ says that $\frac{d\nu}{d\mu},\frac{d\nu}{d\sigma},\frac{d\sigma}{d\mu} $ exist and $\frac{d\nu}{d\mu},\frac{d\nu}{d\sigma},\frac{d\sigma}{d\mu}: X \rarw [0,\infty]$, are $\A$-measurable. Then applying (1) to $\cf{E}\frac{d\nu}{d\sigma}$ for all $E \in \A$, gets $ \int_E \frac{d\nu}{d\sigma} \, d\sigma = \int_E \frac{d\nu}{d\sigma} \frac{d\sigma}{d\mu} \, d\mu$ for all $E \in \A$, so again $ \frac{d\nu}{d\sigma} \frac{d\sigma}{d\mu}  = \frac{d\nu}{d\sigma}$ $\mu$-a.e..\\


\noindent
Proof: 4) $\mu \ll \mu$, $\nu \ll \nu$, and $\frac{d\mu}{d\mu} = 1 = \frac{d\nu}{d\nu}$ $\mu,\nu$-a.e., because for all $E \in \A$, $\mu(E) = \int_E d\mu = \int_E \frac{d\mu}{d\mu} \, d\mu \rimply \frac{d\mu}{d\mu} = 1$ $\mu-$a.e., similarly for $\nu$.  Let $A = \{ x \in X; \frac{d\nu}{d\mu} (x) = 0 \}$, then $\nu(A) = \int_A 0 \, d\mu = 0 \rimply \mu(A) = 0 = \int_A \frac{d\mu}{d\nu} \, d\nu \rimply$ $ \frac{d\mu}{d\nu} = 0$ $\nu$-a.e. on $A$, but $\nu-$a.e. on a $\nu$-null set means everywhere, and similarly, $B = \{ x \in X; \frac{d\mu}{d\nu} (x) = 0 \}$, $\mu(B) = \nu(B) = 0$. So $\frac{d\nu}{d\mu}, \frac{d\mu}{d\nu} > 0$ a.e. Then, let $\{E_k \}_{k \in \nats} \in \A$, $X = \cup_k E_k$, $\nu(E_k) < \infty, \mu(E_k) < \infty$, possible by both $\nu, \mu$ $\sigma$-finite. Then let $\nu_k(E) = \nu(E \cap E_k)$, $\mu_k(E) = \mu(E \cap E_k)$, then $\mu_k(E_k) = \int_{E_k} \frac{d\mu_k}{d\nu_k} \, d\nu_k < \infty \rimply$ $\frac{d\mu_k}{d\nu_k} < \infty$ $\nu_k$-a.e. and so $\mu_k$-a.e. by the finiteness principle, similarly for $\frac{d\nu_k}{d\mu_k}$. So on $E_k$, $\left( \frac{d\mu_k}{d\nu_k} \right) ^{-1}$ and $\left( \frac{d\mu_k}{d\nu_k} \right) ^{-1}$ are well defined, i.e., take either to be 1 on null sets where they are ill-defined, and these won't affect integrals, and hence the associted measures; trying to avoid $1/0$ and $1/\infty$. Then, by (3), $ \nu \ll \mu \ll \nu \rimply \frac{d\nu}{d\nu}= \frac{d\nu}{d\mu}\frac{d\mu}{d\nu} = 1$ a.e., and dividing gives $\frac{d\mu}{d\nu} = \left( \frac{d\nu}{d\mu} \right)^{-1}$.




\end{document}
