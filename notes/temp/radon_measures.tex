
\documentclass[12pt]{article}

\usepackage{amsmath}
\usepackage{amssymb}
\usepackage{amsfonts}

\usepackage{latexsym}
\usepackage{graphicx}
\usepackage{colonequals}
\usepackage{enumerate}

\setlength\topmargin{-1in}
\setlength{\oddsidemargin}{-0.5in}
%\setlength{\evensidemargin}{1.0in}

%\setlength{\parskip}{3pt plus 2pt}
%\setlength{\parindent}{30pt}
%\setlength{\marginparsep}{0.75cm}
%\setlength{\marginparwidth}{2.5cm}
%\setlength{\marginparpush}{1.0cm}
\setlength{\textwidth}{7.5in}
\setlength{\textheight}{10in}


\usepackage{listings}


\newcommand{\pset}[1]{ \mathcal{P}(#1) }
\newcommand{\partset}[1]{ \mathcal{P}^{*}(#1) }

\newcommand{\st}[0]{ \textrm{ s.t. } }
\newcommand{\fall}[0] { \textrm{ for all } }
\newcommand{\wrt}[0] { \textrm{ w.r.t. } }
\newcommand{\aew}[0] { \textrm{a.e.} }
\newcommand{\where}[0] { \textrm{ where } }


\newcommand{\IF}[0] { \; \textrm{if} \; }
\newcommand{\THEN}[0] { \; \textrm{then} \; }
\newcommand{\ELSE}[0] { \; \textrm{else} \; }
\newcommand{\AND}[0]{ \; \textrm{ and } \;  }
\newcommand{\OR}[0]{ \; \textrm{ or } \; }

\newcommand{\rimply}[0] { \Rightarrow }
\newcommand{\limply}[0] { \Leftarrow }
\newcommand{\rlimply}[0] { \Leftrightarrow }
\newcommand{\lrimply}[0] { \Leftrightarrow }

\newcommand{\rarw}[0] { \rightarrow }
\newcommand{\larw}[0] { \leftarrow }

\newcommand{ \defeq }[0] { \colonequals }
\newcommand{ \eqdef }[0] { \equalscolon }
\newcommand{ \cf }[1] { \mathbf{1}_{#1} }

\renewcommand{\Re}{ \operatorname{Re} }
\renewcommand{\Im}{ \operatorname{Im} }

\newcommand{\nats}[0] { \mathbb{N}}
\newcommand{\reals}[0] { \mathbb{R}}
\newcommand{\cmplxs}[0] { \mathbb{C}}
\newcommand{\complexes}[0] { \mathbb{C}}
\newcommand{\scalars}[0] { \mathbb{F}}
\newcommand{\exreals}[0] {  [-\infty,\infty] }

\newcommand{\eps}[0] {  \epsilon }
\newcommand{\om}[0] { \omega }
\newcommand{\Om}[0] { \Omega }

\newcommand{\A}[0] { \mathcal{A} }
\newcommand{\B}[0] { \mathcal{B} }
\newcommand{\C}[0] { \mathcal{C} }
\newcommand{\D}[0] { \mathcal{D} }
\newcommand{\E}[0] { \mathcal{E} }
\newcommand{\F}[0] { \mathcal{F} }
\newcommand{\G}[0] { \mathcal{G} }
\newcommand{\M}[0] { \mathcal{M} }
\newcommand{\cS}[0] { \mathcal{S} }
\newcommand{\U}[0] { \mathcal{U} }
\newcommand{\V}[0] { \mathcal{V} }
\newcommand{\W}[0] { \mathcal{W} }

\newcommand{\Bl}[0] { \mathcal{B} \ell }
\newcommand{\Ell}[0] { \mathcal{L} }

\newcommand{ \Ball } { \textrm{Ball} }
\newcommand{ \Var } { \textrm{Var} }
\newcommand{ \Ker } { \textrm{Ker} }
\newcommand{ \supp } { \textrm{supp} }

\newcommand{ \SUP }[1] { \sup \left( \left\{ #1 \right\} \right) }



\begin{document}

\break

\begin{flushleft}
Radon measures\\
\end{flushleft}

\hrule

\vspace{0.5in}


Call $C_c(X,Y)$ the space of all continuous functions on $X$, to $Y$, if $Y$ is not specified, assume $Y = \reals$. Write $||f||_u$ for the uniform norm of $f$, i.e., $||f||_u = \SUP{ |f(x)|; x \in X }$ A linear functional $I$ on $C_c(X)$ is positive is $I(f) \ge 0$ when $f \ge 0$. If $U$ is open in $X$, and $f \in C_c(X)$, write $f \prec U$ to mean that $f(x) \in [0,1]$ and $\supp(f) \subset U$, this is a stronger statement than $0 \le f \le \cf{U}$, which only implies that $\supp(f) \subset \overline{U}$. \\



Proposition (Folland 7.1), if $I$ is a positive linear functional on $C_c(X)$, for each compact $K \subset X$, there is a constant, $C_K$, such that $|I(f)| \le C_K ||f||_u$ for all $f \in C_K(X)$ such that $\supp(f) \subset K$. \\

If $\mu$ is a Borel measure on $X$ such that $\mu(K) < \infty$ for every compact $K \subset X$, then $C_c(X) \subset L^1(\mu)$, so that $ f \mapsto \int f \, d\mu$ is a positive linear functional on $C_c(X)$. \\



If $I$ is a positive linear functional on $C_c(X)$, there is a unique Radon measure, $\mu$ on $\Bl(X)$ such that $I(f) = \int_X f \, d\mu$ for all $f \in C_c(X)$. Moreover , $\mu$ satisfies

\begin{enumerate}[a)]
\item
$\mu(U) = \sup \left( \{ I(f); f\in C_c(X), f \prec U  \} \right) $ for all open $U \subset X$
\item
$\mu(K) = \inf \left( \{ I(f); f \in C_c(X), f \ge \cf{K} \}\right) $ for all compact $K \subset X$.
\end{enumerate}

\noindent
Proof: ADD






















\end{document}
