
preamble,file:
    ../preamble.tex

macros:
    file:
        ../macros.tex
    
    \newcommand{\A}[0] { \mathcal{A} }
    \newcommand{\B}[0] { \mathcal{B} }
    \newcommand{\C}[0] { \mathcal{C} }
    \newcommand{\D}[0] { \mathcal{D} }
    \newcommand{\E}[0] { \mathcal{E} }
    \newcommand{\F}[0] { \mathcal{F} }
    \newcommand{\G}[0] { \mathcal{G} }
    \newcommand{\M}[0] { \mathcal{M} }
    \newcommand{\N}[0] { \mathcal{N} }
    \newcommand{\curlyO}[0] { \mathcal{O} }
    \newcommand{\R}[0] { \mathcal{R} }
    \newcommand{\curlyS}[0] { \mathcal{S} }
    \newcommand{\U}[0] { \mathcal{U} }
    \newcommand{\V}[0] { \mathcal{V} }
    \newcommand{\W}[0] { \mathcal{W} }
    \newcommand{\Bl}[0] { \mathcal{B} \ell }
    \newcommand{\Ell}[0] { \mathcal{L} }       


document:
    title:
        Topology, Fall 2010
    author:
        Dr. Blecher
    
    section:
        name:
            Topology
        number:
            1.2
    
    definition:
        number:
            1.2.1
        
        A topology on a set $X$ is a collection $\tau \in \pset{X}$ such that:
        
        enumerate:
            token:
                i
            
            item:
                $\tau$ is closed under arbitrary unions
            item:
                $\tau$ is closed under finite intersections
            item:
                $\phi, X \in \tau$.
        
        We say $(X, \tau)$ is a topological space. A subset $U \subset X$ is called open iff $U \in \tau$. A subset $C \subset X$ is called closed iff $C^c \in \tau$.
        
    example:
        enumerate:
            token:
                i
            
            item:
                Discrete toplogy: $\tau = \pset{X}$, empty set is clopen (closed and open)
            item:
                Indiscrete topology: $\tau = \{ \phi, X \}$
            item:
                $X \defeq \{ 1 \}$ has one topology, $\{ \phi, \{ 1\} \}$ (discrete and indecrete are the same). $X \defeq \{1, 2\}$ has four topologies, discrete, indiscrete, $\{ \phi , \{1,2\}, \{1\}\}$, and $\{ \phi, \{1,2\}, \{2\}\}$.
            item:
                $(\reals^n, \tau)$, with $\tau = \{ U \subset X; \; \forall x \in U \; \exists \eps>0 \st B(x,\eps) \subset U \}$. Here $B(x, \eps) = \{ y \in \reals^n; ||x-y|| < \eps \}$.

    definition:
        For two topologies, $\sigma, \tau$ on a set $X$, we say $\sigma$ is finer than $\tau$ if $\sigma \supset \tau$, in this case we also say $\tau$ is coarser than $\sigma$.

    definition:
        number:
            1.2.2
        Let $(X, \tau)$ be a topological space. A subset $\B \subset \tau$ is called a basis for $\tau$, if for all $x \in X$, for all $\tau \ni U \ni x, \; \exists B \in \B \st x \in B \subset U$.
    
    example:
        enumerate:
            token:
                i
            
            item:
                $\B = \tau$, (not powersert; $\B \subset \tau$)
            item:
                $X = \reals^n, \tau$ as in example 4, then $\B = \{ B(x, r); \; x \in X, r>0 \}$.
    
    lemma:
        name:
            1
        
        $(X, \tau)$ a topological space, with basis $\B$. $U \in \tau$, the following are equivalent
        
        enumerate:
            token:
                i
            
            item:
                $U \in \tau$ ($U$ is open)
            item:
                $U$ is a union of elements in $\B$.
            item:
                For all $x \in U$, $\exists B \in \B \st x \in B \subset U$.
        
        proof:
            (2 $\rimply$ 1) Since $\B \subset \tau$, and 1 in definition of a topology.
            (1 $\rimply$ 3) By definition of a basis.
            (3 $\rimply$ 2) By 3, for all $x \in U$  there exists a set $B_x \in \B \st x \in B_x \subset U \rimply U = \cup_{x \in U} \{x\} \subset \cup_{x \in U} B_x \subset \cup_{x \in U} U = U $ $\rimply U = \cup_{x \in U}$. $ \Box$
    
    Notice, that if $\B$ is a basis for a topology $\tau$ on a set $X$, then we have:
    
    list:
        token:
            
        item:
            (B1) for all $x \in X \; \exists B \in \B \st x \in B$ \\
            (B2) for all $B_1, B_2 \in \B$ and for all $x \in B_1 \cap B_2$ $\exists B_3 \in \B \st x \in B_3 \subset B_1 \cap B_2$.
    
    proof:
        list:
            token:
                
            item:
                (B1) follows from the definition of a basis, with $U = X$ \\
                (B2) follows from the definition of a basis, with $U = B_1 \cap B_2$
    
    lemma:
        name:
            2
        
        Conversely, if $X$ is any set, $\B \subset \pset{X}$, with $\B$ satisfying (B1) and (B2) above, then there exists a unique topology $\tau$ such that $\B$ is a basis for $\tau$, uniquely.

        proof:
            (uniqueness) this follows from lemma 1 (ii), the topology is forced to be the union of sets in $\B$. \\
            (existance) Define $\tau = \{ U \subset X; \; \forall x \in U \; \exists B \in \B \st x \in B \subset U \}$. $\phi \in \tau$ trivially, $X \in \tau$ by (B1). If $U_i \in \tau, \; \forall i \in I$, then $\forall x \in \cup_{i \in I} U_i \rimply \exists j \in I \st x \in U_j  \underset{\tau}{\overset{\textrm{def of} }{\rimply}} \exists B \in \B \st$ $ x \in B \subset U_i \subset \cup_{i \in I} U_i \rimply $ $ \cup_{i \in I} U_i \in \tau$, so $\tau$ is closed under arbitrary unions. Let $U, V \in \tau, x \in U \cap V$ by def of $\tau$, $\exists B_1, B_2 \in \B \st x \in B_1 \subset U, x \in B_2 \subset V$, so $x \in B_1 \cap B_2$, so by (B2) $\exists B_3 \in \B \st$ $ x \in B_3 \subset B_1 \cap B_2 \subset U \cap V \rimply$ $ U \cap V \in \tau$, so $\tau$ is closed under finit intersections. Finally, $\B$ is a (unique ?) basis for $\tau$, by the definition of $\tau$ and what a basis is. $\Box$

    example:
        
        enumerate:
            
            item:
                Let $\B = \{ (a,b); a,b \in \reals, a<b \}$
            item:
                Let $\B = \{ B(x, \eps); x \in \reals, \eps>0 \}$, check (B2) in a more general setting.
            item:
                Metric spaces.
            item:
                Normed (vector) spaces.
            
    proposition:
        name:
            1

        Every normed space is a metric space with respect to $d = x,y \mapsto ||x-y||$.

        proof:
            (i) $ d(x,y)=0 \lrimply ||x-y||=0 \lrimply x-y = 0 \lrimply x=y$ (ii) $d(x,y) = ||x-y|| = ||(-1)(y-x)|| = ||y-x|| = d(y,x)$, (iii) $d(x,z) = ||x-z|| = ||(x-y)+(y-z)|| \le ||x-y|| + ||y - z|| = d(x,y) + d(y,z)$.

    proposition:
        name:
            2
            
        $\forall y \in B(x,r) \; \exists s > 0 \st B(y,s) \subset B(x,r)$
            
        proof:
            Take $s = r - d(x,y)$, then for all $z \in B(y, s)$, $d(x,z) \le d(x,y) + d(y,z) < d(x,y) + s =$ $ d(x,y) + r - d(x,y) = r$, so $d(x,z) < r \rimply z \in B(x,r) \rimply B(y,s) \subset B(x,r)$.

            
    proposition:
        name:
            3
            
        $\{ B(x, r); x \in X, r>0 \}$ forms a basis for a topology on the metric space $(X, d)$.
            
        proof:
            We have to check (B1), (B2). (B1) is trivial, because $x \in B(x,r)$. (B2): if $ z \in B(x,\eps) \cap B(y,\delta)$ by prop 2 $\exists r>0 \st B(z,r) \subset B(x,\eps)$, similarly, $\exists s>0 \st B(z,s) \subset B(y,\delta)$. Then $z \in B(z, \min(r,s)) \subset B(x,\eps) \cap B(y,\delta)$ so (B2) holds, so we are done by lemma 2.
        
    discussion:
        Summary: Every metric space $(X, d)$ has an associated topology, called the \emph{metric topology}, write this as $\tau_d$. It has as basis the set of open balls $\{ B(x, r); x \in X, r>0 \}$. Putting this together with prop 1, every normed space, $(X, || \cdot ||)$, has an associated topology, called the \emph{norm topology}, it is $\tau_d$, where $d(x,y) = ||x-y||$.
        
    definition:
    
    A topological space $(x, \tau)$ is called \emph{meterizable} if $\tau = \tau_d$, for some metric $d$ on $X$. Because metric spaces are `nice' or `understood', in many cases it is of interest to show that a topology is meterizable; we will mention some tests later.
    
    proposition:
        name:
            4
        
        $U \in \tau_d \lrimply \forall x \in U \; \exists r>0 \st B(x,r) \subset U$.

        proof:
            Use the definition of $\tau$ in the proof of lemma 2, also use prop 3.

    discussion:
    
        Applying all of this to the usual euclidian metrix in $\reals^n$, the $\tau_d$ we get is precicely what we call the `open sets' in undergraduate analysis.
        
    lemma:
        number:
            1.2.4
        name:
            1
        
        If $\B_i$ is a basis for a topology $\tau_i$ on $X$, $i \in \nats$, $i \in \{1,2\}$, then $\tau_1 \subset \tau_2 \lrimply \forall x \in X, \forall B \in \B_1, \exists C \in B_2 \st x \in C \subset B$.


        proof:
            ($\limply$) if $U \in \tau_1, x \in U$, then $\exists B \in \B$ such that $x \in B \in U$. By hypothesis $\exists C \in \B_2 \st x \in C \subset B \subset U$. So $U \in \tau_2$, by lemma 1.
        
    { \bf skipped some things here; left off at page 6, picking up at page 9.}

    
    
    
    
    
    
    
    definition:
        number:
            1.2.6
        
        A \emph{neighborhood} of a point $x$ in a topological space $(X, \tau)$ is an open set $U \in \tau$ containing $x$ (some authors sefine a neighborhood as any set containing $x$). The \emph{neighborhood basis} of $x$, written $\curlyO(x)$ is $\{ U \in \tau; x \in U\}$.
    
    
    definition:
        number:
            1.2.7
        We say $A \subset X$ is \emph{closed} if $A^c$ is open.
        
    proposition:
        enumerate:
            item:
                Arbitrary intersections of closed sets are closed.
            item:
                Finite unoins of closed sets are closed.
            item:
                $\phi, X$ are closed.
        proof:
            By De Morgan's laws

    example:
        In a metric space, $(X, d)$, $\overline{B}(x, \eps) = \{ y \in X; d(x,y) \le \eps \}$ is closed in the metric topology $\tau_d$.
    
        
        
    definition:
        number:
            1.2.8
        
        The \emph{closure}, $\overline{A}$, of $A$, is the intersection of all closed subsets of $X$, containing $A$.

        Note, $\overline{A}$ is the smallest closed subset of $X$ containing $A$. $A$ closed $\lrimply A = \overline{A}$.
        
        
    proposition:
        name:
            1
        enumerate:
            item:
                $x \in \overline{A} \lrimply U \cap A = \phi \; \forall x \in U \in \tau$.
            item:
                If $\tau$ has a basis $\B$ then $x \in \overline{a} \lrimply U \cap A \not = \phi, \forall x \in U \in \B$.
            item:
                In a metric space $(X, d)$, $x \in \overline{A} \lrimply B(x, r) \cap A \not = \phi \; \forall r>0$.
        
        proof:
            to do

    definition:
        The \emph{interior}, $A^o$ or $\textrm{int}(A)$ is the union of all upen sets contained inside $A$, which also equals the biggest open set inside $A$. Clearly $A$ open $\lrimply A = A^o$.

    proposition:
        name:
            2
        enumerate:
            item:
                $x \in A^o \lrimply \; \exists U \in \tau \st x \in U \subset A$
            item:
                If $\tau$ has a basis $\B$ then $x \in A^o \lrimply B \in \B \st x \in B \subset A$
            item:
                In a metric space, $x \in A^o \lrimply \; \exists \eps > 0 \st B(x, \eps) \subset A$

        proof:
            Ex.
    

    proposition:
        name:
            3
        For all $A, B \subset X$
        enumerate:
            item:
                $A \subset B \rimply \overline{A} \subset \overline{B}$, and $A^o \subset B^o$.
            item:
                $\overline{A \cup B} = \overline{A} \cup \overline{B}, ( A \cup B )^o \supset A^o \cup  B^o$
            item:
                $\overline{A \cap B} \subset \overline{A} \cap \overline{B}$ and $ (A \cap B)^o = A^o \cap B^o$.
            item:
                $(\overline{A})^c = (A^c)^o$ and $(A^o)^c = \overline{A^c}$
        proof:
            Ex.
    
    definition:
        number:
            1.2.9
        Boundary and accumulation points: $\textrm{bdy}(A) = \partial A \defeq \overline{A} \setminus A^o$, $\overline{\phi} = \phi, \overline{x} = X$.
        
        An accumulation (or limit or cluster) point of $A$ is an element $x \in X$ such that every neighborhood of $x$ contains at least one point in $A \setminus \{ x \}$.
        
    proposition:
        
        For a non-empty set $A$ in a topological space $(X, \tau)$:
        
        enumerate:
            token:
                i
            item:
                $x \in \partial A \lrimply \forall $ open $U$ comtaining $x$, $U \cap A \not = \phi$, and $U \cap A^c \not = \phi$
            item:
                $\partial A = \partial (A ^c) = \overline{A} \cap \overline{A^c}$
            item:
                $\overline{A} = A \cup \partial A = A^o \cup \partial A$, $a^o = A \setminus \partial A = \overline{A} \setminus \partial A$
            item:
                $A$ open $\lrimply A \cap \partial A = \phi$, $A$ closed $\lrimply \partial A \subset A$.
            
        Let $A'$ be the set of accumulation points of $A$, By (i), $x \in A' \lrimply x \in \overline{ A \setminus \{ x \}}$. As we saw in undergraduate analysis, $\partial A \not \subset A'$ nor $A' \not \subset \partial A$, in general.


    definition:
        number:
            1.2.10
        $(X, \tau)$ a topological space and $Y$ a subset of $X$, the \emph{subspace} or \emph{relative topology} on $Y$ is $\tau_Y = \{ U \cap Y; U \in \tau \}$. This is called the \emph{topology induced on $Y$ from/by $X$}. Sets in $\tau_Y$ are called `relatively open' sets. We also say that $Y$ with its topology $\tau_Y$ is a \emph{subspace} of $X$. Similarily, a subset $A$ of $Y$ is \emph{relatively closed}, or \emph{closed in subspace topology} if $Y \setminus A$ is relatively open (= open in subspace topology).
    
        proof:
            We need to show that $\tau_Y$ is a topology. $\phi = \phi \cap Y \in \tau_Y$, $Y = X \cap Y \in \tau_Y$, let $\{ U_1, ..., U_n \} \in \tau_Y$, then $\cap_{k=1}^n (U_k \cap Y) = \left( \cap_{k=1}^n U_k \right) \cap Y \in \tau_Y$, since $\cap_{k=1}^n U_k \in \tau$. Finally, if $\{ U_i; i \in I \} \subset \tau_Y$, then $\cup_{i \in I} ( U_i \cap Y) = \left( \cup_{i \in I} U_i \right) \cap Y \in \tau_Y$ since $\cup_{i \in I} U_i \in \tau$.
    
    example:
        $[0,1)$ is an open set in $[0,2]$ if $[0,2]$ has a subspace topoogy. Because $[0,1) = [0,2] \cap (-1,1)$. It is relatively open in $[0,2]$. 
    

    lemma:
        enumerate:
            item:
                If $A \subset Y \subset (X, \tau)$, then $A$ is relatively closed in $Y \lrimply A = Y \cap C, C$ closed in $X$.
            item:
                A subspace of a subspace of $(X, \tau)$ is a subspace of $(X, \tau)$.
            item:
                If $\B$ is a basis for $\tau$, on $X$, then $\{ U \cap Y; U \in \B \}$ is a basis for $\tau_Y$.
            item:
                If $A \subset Y \subset (X, \tau)$, then the closure of $A$ with respect to $\tau_Y$ equals $\overline{A} \cap Y$, where $\overline{A}$ is the closure with respect to $\tau$. 
            item:
                not finished.



    {\bf left off on page 13, picking up on page 16/17}
    
    section:
        number:
            1.3
        name:
            Convergence
    
    
    definition:
    First, let's discuss orderings on a set. A \emph{Binary relation} on $X$ is a subset $\R \subset X \times X$. We write $x \le y$ iff $(x,y) \in \R$. A \emph{Preordered set} is a set $X$ with a binary relation, $\le$ satisfying
    
    enumerate:
        token:
            i
        item:
            Reflexivity: $x \le x \; \forall x \in X$\
        item:
            Antisymmetry: $x \le y, y \le x \rimply x=y$
        item:
            Transitivity: $x \le y, y \le z \rimply x \le z$.
    
    A \emph{Partially ordrerd set} (poset) is a set which is preordred and antisymmetric, $x \le y \le x \rimply x=y$.
    
    A subset $Y$ of a preordered set $X$ is said to have an \emph{upper bound} in $X$ if $\exists \; x \in X \st y \le x \forall y \in Y$.
    
    A \emph{directed set} is a non-empty preordered set $X$ such that $\{x,y\}$ has an upper bound in $X$ for all $x,y \in X$.
    
    A \emph{totally ordered set} is a poset such that either $x \le y$ or $y \le x$ for all $x,y \in X$.
    
    A \emph{well ordered set} is a poset such that every non-empty $Y \subset X$ has a \emph{minimum:} $\exists y \in Y \st y \le x \; \forall x \in Y$.
    












