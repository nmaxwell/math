\documentclass[12pt]{article}

\usepackage{amsmath}
\usepackage{amssymb}
\usepackage{amsfonts}

\usepackage{latexsym}
\usepackage{graphicx}
\usepackage{colonequals}
\usepackage{enumerate}

\setlength\topmargin{-1in}
\setlength{\oddsidemargin}{-0.5in}
%\setlength{\evensidemargin}{1.0in}

%\setlength{\parskip}{3pt plus 2pt}
%\setlength{\parindent}{30pt}
%\setlength{\marginparsep}{0.75cm}
%\setlength{\marginparwidth}{2.5cm}
%\setlength{\marginparpush}{1.0cm}
\setlength{\textwidth}{7.5in}
\setlength{\textheight}{10in}


\usepackage{listings}




\newcommand{\pset}[1]{ \mathcal{P}(#1) }
\newcommand{\partset}[1]{ \mathcal{P}^{*}(#1) }
\newcommand{\st}[0]{ \textrm{ s.t. } }
\newcommand{\fall}[0] { \textrm{ for all } }
\newcommand{\wrt}[0] { \textrm{ w.r.t. } }

\newcommand{\nats}[0] { \mathbb{N}}
\newcommand{\reals}[0] { \mathbb{R}}
\newcommand{\cmplxs}[0] { \mathbb{C}}
\newcommand{\complexes}[0] { \mathbb{C}}
\newcommand{\exreals}[0] {  [-\infty,\infty] }
\newcommand{\eps}[0] {  \epsilon }
\newcommand{\A}[0] { \mathcal{A} }
\newcommand{\B}[0] { \mathcal{B} }
\newcommand{\C}[0] { \mathcal{C} }
\newcommand{\D}[0] { \mathcal{D} }
\newcommand{\E}[0] { \mathcal{E} }
\newcommand{\F}[0] { \mathcal{F} }
\newcommand{\G}[0] { \mathcal{G} }
\newcommand{\M}[0] { \mathcal{M} }
\newcommand{\cS}[0] { \mathcal{S} }

\newcommand{\om}[0] { \omega }
\newcommand{\Om}[0] { \Omega }

\newcommand{\Bl}[0] { \mathcal{B} \ell }

\newcommand{\Ell}[0] { \mathcal{L} }


\renewcommand{\Re}{ \operatorname{Re} }
\renewcommand{\Im}{ \operatorname{Im} }

\newcommand{\IF}[0] { \; \textrm{if} \; }
\newcommand{\THEN}[0] { \; \textrm{then} \; }
\newcommand{\ELSE}[0] { \; \textrm{else} \; }
\newcommand{\AND}[0]{ \; \textrm{ and } \;  }
\newcommand{\OR}[0]{ \; \textrm{ or } \; }

\newcommand{\rimply}[0] { \Rightarrow }
\newcommand{\limply}[0] { \Lefttarrow }
\newcommand{\rlimply}[0] { \Leftrightarrow }
\newcommand{\lrimply}[0] { \Leftrightarrow }

\newcommand{\rarw}[0] { \rightarrow }
\newcommand{\larw}[0] { \leftarrow }

\newcommand{ \defeq }[0] { \colonequals }
\newcommand{ \eqdef }[0] { \equalscolon }
%\newcommand{ \cf }[1] { \chi_{_{#1}} }
\newcommand{ \cf }[1] { \mathbf{1}_{#1} }
\newcommand{ \Var } { \textrm{Var} }

%\newcommand{\Re}[0] { \textrm{ Re } }
%\newcommand{\Re}[0]{ \textrm{Re} }
%\newcommand{\Im}[0]{ \textrm{Im} }

\begin{document}

\begin{flushleft}
Notes on Integration and differentiation on $\reals^n$, etc. \\
Nicholas Maxwell\\
\end{flushleft}

\begin{flushleft}
\addvspace{5pt} \hrule
\end{flushleft}	

A measure defined on a Borel $\sigma$-algebra is called a Borel measure. Write $\M(\reals^n)$ for the Borel measures on $\reals^n$. If $E \subset \reals^n$, then $\M(E)$ are the borel measures on $E$, but $\M(E) \subset \M(\reals^n)$, by zero padding. \\

Definition: a positive finite measure $\mu$ on $\reals^n$ is regular if

\begin{enumerate}
\item
$\mu(E) = \inf \{ \mu(U) ; U \textrm{ open}, U \supset E \}$
\item
$\mu(E) = \sup \{ \mu(K) ; K \textrm{ compact}, K \subset E \}$
\end{enumerate}

Every positive finite measure on $\reals^n$ is regular. For every measure $\nu \in \M(\reals^n), \fall E \in \Bl(\reals^n)$, there exists a sequence of open sets $U_k \supset E$, and compact sets $K_n \subset E$ such that $\nu(U_n) \rarw \nu(E)$ and $\nu(K_n) \rarw \nu(E)$. \\

\noindent
Proof: ADD \\

Definition: we say a measure $\nu \in \M(\reals^n)$ is regular if each positive $\nu_k$, in the Jordan decomposition $\nu = \sum_{k=0}^3 i^k \, \nu_k$, is regular. By the last result, every $\nu \in \M(\reals^n)$ is regular and then the condition about sequences of sets holds. \\

If $\nu \in \M(\reals)$, we define its distribution function by $F_\nu (x) = \nu((\infty,x])$. $\nu \mapsto F_\nu$ is injective and linear on $\M(\reals)$. \\

\noindent
Proof: ADD \\

Def: $F: \reals \rarw \reals$ is of bounded variation, BV, or say $F \in BV$, if $\Var(F) < \infty$, where $\Var(F) \defeq \sup\{ V_F(x); x \in \reals \}$, and 

$$
V_F(x) \defeq \sup \left\{  \sum_{k=1}^n |F(x_k) - F(x_{k-1})| ; \;  x_0    < x_1 < ... < x_n=x, ( x_k ) \in \reals, n \in \nats \right\}.
$$

Def: $F: \reals \rarw \reals$ is in $NBV$ if $F \in BV$, $F$ is right continuous at all $x \in \reals$, and $\lim_{x \rarw -\infty} F(x) = 0$; normalized $BV$.\\

If $\nu \in \M(\reals, \Bl(\reals))$, then $F_\nu \in NBV$. \\

\noindent
Proof: ADD \\


(Folland 3.28) If $F \in BV$ then $\lim_{x \rarw - \infty} V_F(x) = 0$ and $F \in BV \rimply V_F \in NBV$.

\noindent
Proof: ADD \\


Properties of $BV$,

\begin{enumerate}[1)]
\item
If $F,G: \reals \rarw \reals$, $c \in \reals$, then $V_{F+G}(x) \le V_F(x) + V_F(x)$ and $V_{cF}(x) = |c| V_F(x)$. Hence $BV$ is a vector space and if $F,G \in BV$, then $\Var(F+G) \le \Var(F) + \Var(G)$ and $\Var(cF) = |c| \Var(F)$. $NBV$ is a subspace of $BV$.
\item
If $F \in BV$, then $V_F(x)$ is an increasing function of $x$, bounded above by $\Var(F)$.
\item
\begin{enumerate}[a)]
\item
Moreover, if $x < y$, then $V_F(y)-V_F(x) = \sup \left( \left\{ \sum_{k=1}^n | F(x_k) - F(x_{k-1}) | ; x \le x_0 < x_1 < ... < x_{n} = y \right\} \right)$.
\item
special capse: $F(y) - F(x) \le V_F(y) - V_F(x) \le V_F(y) \le \Var(F)$.
\item
consequence: $F \in BV \rimply F$ is bounded.
\end{enumerate}
\item
An increasing $F: \reals \rarw \reals$ is in $BV$ iff $F$ is bounded.
\item
$F: \reals \rarw \reals \in BV$ iff $F =  F_1-F_2$, where $F_1,F_2: \reals \rarw \reals$ are bdd and increasing.
\item
$F: \reals \rarw0 \complexes \in BV$ iff $\Re F, \Im F \in BV$.
\item
$F \in BV \rimply F$ continuous except at countable many points, and for all $x \in \reals$, $F(x+) = \lim_{t \rarw x^+} F(t)$ and $F(x-) = \lim_{t \rarw x^-} F(t)$, and $\lim_{x \rarw +\infty} F(x)$ and $\lim_{x \rarw -\infty} F(x)$ all exist and are in $\reals$.
\item
 $F \in BV \lrimply F = F_1 - F_2 + i F_3 - i F_4$, where $F_k : \reals \rarw \reals$, increasing, bounded, right continuous, and $\lim_{x \rarw - \infty} F_k(x) = 0$ for all $k$.
\end{enumerate}


\noindent
Proof: ADD \\


The linear map $T = \nu \mapsto F_\nu$ from $\M(\reals)$ to $NBV$ is an isomorphism. Thus it is bijective and $\Var(F_\nu) = ||\nu||$ for all $\nu \in \M(\reals)$, which implies that $NBV$ is a Banach space with norm $||F|| = \Var(F)$, $||T(\nu)|| = ||\nu||$. \\



\noindent
Proof: ADD \\









\end{document}








